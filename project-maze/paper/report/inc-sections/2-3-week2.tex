\subsubsection{2주차: 미로를 화면에 그리는 로직} 미로를 화면에 그리기 위해 미로를 2차원 동적 배열에 저장한다.

이 때 각 칸은 `$+$', `$-$', `$|$' 또는 `.'으로 구성되는데, `.'의 경우 길이 있음을, `$+$', `$-$', `$|$'의 경우 벽이 있음을 나타내므로
이에 따라 적절한 크기의 사각형을 적절한 위치에 그리면 될 것이다. 따라서 $rows \times cols$의 미로의 경우 직사각형을 $rows \times cols$번 그리는데,
지금 확인하는 칸이 `+'일 경우 작은 정사각형을, `-'일 경우 좌우로 긴 직사각형을, `$|$'일 경우 상하로
긴 직사각형을 적절한 위치에 그린다. 크기는 \texttt{WALL_WIDTH}와 \texttt{CELL_SIZE}에 미리 정의한다.

$i$와 $j$가 홀수일 경우는 현재 확인하고 있는 칸은 원래 미로의 각 칸이 되는 경우이고 $i$ 혹은 $j$가 짝수일 경우는 원래 미로의 벽 혹은 통로가 되는 경우이다.
따라서 이 점을 이용해 직사각형을 그리기 시작하는 왼쪽 위 $x$ 좌표는 $i$에 대해
\[\left(CELL\_SIZE \times \left\lfloor\frac{i}{2}\right\rfloor\right) + \left(WALL\_WIDTH \times \left\lceil\frac{i}{2}\right\rceil\right)\]
이고, 이는 $y$ 좌표에 대해서도 마찬가지이다. 직사각형을 그리는 작업이 상수 시간만큼 걸린다고 가정하면 미로를 그리는 데 소요되는 시간 복잡도는 $\mathcal{O}\left(rows \times cols\right)$
이고 미로를 저장하는 공간 복잡도도 $\mathcal{O}\left(rows \times cols\right)$이다.

특히 `.' 칸들에만 정점이, 인접한 `.' 칸들 사이에만 간선이 있다고 생각하면
미로를 그래프처럼 생각하고 탐색할 수 있으므로 굳이 가로/세로 길을 계산해 저장할 필요가 없다.

\subsubsection{2주차: 툴바에 버튼 추가} MFC API를 사용하여 툴바에 버튼을 추가한다. 프로젝트의 리소스는 \texttt{MFC_Main.rc}라는 파일이 관리하는데,
프로젝트의 \texttt{IDR\_MAINFRAME}에서 Event Handler Wizard를 이용해 메뉴를 추가할 수 있다.