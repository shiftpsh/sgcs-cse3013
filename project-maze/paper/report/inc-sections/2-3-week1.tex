\subsubsection{1주차: 완전 미로를 생성하는 로직} 높이 $h$, 너비 $w$의 미로에서 백트래킹은 다음과 같이 작동한다.

\begin{codebox}
\Procname{\proc{Backtrack}($x$, $y$, $h$, $w$)}
\li Calculate unvisited adjacent cells to $(x,\,y)$
\li Set visited state of $(x,\,y)$ to \const{true}
\li \While there exists unvisited adjacent cells to $(x,\,y)$ \Do
\li     $(\id{nx},\,\id{ny}) \gets$ random unvisited adjacent cell to $(x,\,y)$
\li     Make path from $(x,\,y)$ to $(\id{nx},\,\id{ny})$
\li     \proc{Backtrack}(\id{nx}, \id{ny}, $h$, $w$)
\li     Recalculate unvisited adjacent cells to $(x,\,y)$
    \End
\end{codebox}

랜덤하게 결정한 $(x,\,y)$에서 백트래킹을 시작하면 된다.

자료구조는 가로 길을 나타내는 $h \times \left(w - 1\right)$ 크기의 불 배열과 세로 길을 나타내는 $\left(h - 1\right) \times w$ 크기의
불 배열로 표현하였다. 실제로는 각각 \texttt{horizontal_route}와 \texttt{vertical_route}라는 이름으로 구현되었는데,
미로에서 $(x,\,y)$에서 $(x,\,y-1)$로 갈 수 있다면 \texttt{vertical_route[y - 1][x]}가 \texttt{1}이고, 아닐 경우 \texttt{0}인 자료구조이다.
이 자료구조의 공간 복잡도는 $\mathcal{O}\left(h\times w\right)$이다.

인접한 셀의 개수를 계산하는 알고리즘은 4방향만 체크하면 되므로 $\mathcal{O}\left(1\right)$으로 충분하다.
백트래킹은 각 셀마다 인접한 미방문 셀을 방문하는데, 결국 모든 셀을 방문하게 되므로 시간 복잡도도 $\mathcal{O}\left(h\times w\right)$이다.

\subsubsection{1주차: 불완전 미로를 생성하는 로직} 불완전 미로의 생성은 생성된 완전 미로에서 추가로 몇 갸의 벽을 없애 주는 것으로 가능하다.
모든 칸들을 랜덤하게 돌면서 옆 셀과 벽으로 막혀 있는 셀이 있다면 길을 만들어 주는 식으로 $n$번 반복한다. 자료구조는 그대로 사용하였다.

\begin{codebox}
\Procname{\proc{Delete-Walls}(\id{delete}, $h$, $w$)}
\li $\id{delete} \gets \min\left(\id{delete},\,\frac{\min\left(h,\,w\right)}{2}\right)$
\li \While $\id{delete} > 0$ \Do
\li     $(x,\,y) \gets$ random cell in maze field
\li     $(\id{nx},\,\id{ny}) \gets$ random adjacent cell to $(x,\,y)$
\li     \If there is no direct path between $(x,\,y)$ and $(\id{nx},\,\id{ny})$ \Then
\li         Make new direct path
\li         $\id{delete} \gets \id{delete} - 1$
        \End
    \End
\end{codebox}

정점이 $hw$개 존재하는 그래프의 최소 신장 트리에는 노드가 $hw-1$개 존재한다. 또한 미로의 모든 벽을 삭제하면 간선은
$h \times \left(w - 1\right) + \left(h - 1\right) \times w$개 존재한다. 따라서 완전 미로 상태에서 셀과 셀 사이의
벽은
\begin{align*}
    & \left[h \times \left(w - 1\right) + \left(h - 1\right) \times w\right] - \left[hw-1\right] \\
    =&\, hw - h - w + 1 \\
\end{align*}
개이다. 따라서 총 $h \times \left(w - 1\right) + \left(h - 1\right) \times w$개의 원래 존재하던 벽 중 $hw - h - w + 1$개를
고르는 확률로 벽을 지우는 데 성공할 수 있다.

$n$개의 벽 중 $k$개의 벽이 남아 있을 때, 지우기에 성공하는 확률은 $\dfrac{k}{n}$이다. 따라서 이 때 처음으로 존재하는 벽을 지우기까지의 시도 횟수를
$X$라 하면 $X \sim \mathrm{Exp}\left(\dfrac{k}{n}\right)$이므로 이 때 $\mathrm{E}\left(X\right) = \dfrac{n}{k}$이다.

이 알고리즘에서 벽을 $r$개 지우도록 한다면 벽 $r$개를 지우는 데 필요한 시도 횟수의 기댓값은
$n$번째 조화수 $\displaystyle H_n = \sum_{k=1}^n \dfrac{1}{k}$라고 할 때
\begin{align*}
    & \sum_{k=0}^{r-1} \dfrac{h \times \left(w - 1\right) + \left(h - 1\right) \times w}{hw - 1 - k} \\
    =& \sum_{k=0}^{r-1} \dfrac{2hw - h - w}{hw - 1 - k} \\
    =& \left(2hw - h - w\right) \left(H_{h\times w - 1} - H_{h\times w - r - 1}\right)
\end{align*}
이다.

한편 상수 $\gamma \approx 0.577$에 대해 $\displaystyle \lim_{n \rightarrow \infty} \left(H_n - \ln n\right) = \gamma$임이 알려져 있으므로,
$hw = A \rightarrow \infty$일 때

\begin{align*}
    & \lim_{A\rightarrow\infty} \sum_{k=0}^{r-1} \dfrac{h \times \left(w - 1\right) + \left(h - 1\right) \times w}{hw - 1 - k} \\
    =& \lim_{A\rightarrow\infty} \left[\left(2A - h - w\right) \left(H_{A - 1} - H_{A - r - 1}\right)\right] \\
    =& \lim_{A\rightarrow\infty} \left[\left(2A - h - w\right) \left(\ln \left(A - 1\right) - \ln \left(A - r - 1\right)\right)\right] \\
    =& \lim_{A\rightarrow\infty} \left[\left(2A - h - w\right) \ln \frac{A - 1}{A - r - 1}\right] \\
    =& \lim_{A\rightarrow\infty} \ln \left(\frac{A - 1}{A - r - 1}\right)^{2A - h - w} \\
    =& \lim_{A\rightarrow\infty} \ln \left(1 + \frac{r}{A - r - 1}\right)^{2A - h - w} \\
    =& \lim_{A\rightarrow\infty} \ln \left(1 + \frac{r}{A - r - 1}\right)^{2r \times \frac{A - r - 1}{r} + 2r + 1 - h - w} \\
    =& \lim_{A\rightarrow\infty} \ln \left[\left[\left(1 + \frac{r}{A - r - 1}\right)^{\frac{A - r - 1}{r}}\right]^{2r} \left(1 + \frac{r}{A - r - 1}\right)^{2r + 1 - h - w}\right] \\
    =& \lim_{A\rightarrow\infty} \ln \left[\left(1 + \frac{r}{A - r - 1}\right)^{\frac{A - r - 1}{r}}\right]^{2r} + \lim_{A\rightarrow\infty} \ln \left(1 + \frac{r}{A - r - 1}\right)^{2r + 1 - h - w} \\
    =& \ln \lim_{A\rightarrow\infty} \left[\left(1 + \frac{r}{A - r - 1}\right)^{\frac{A - r - 1}{r}}\right]^{2r} + \left(2r + 1 - h - w\right) \ln \lim_{A\rightarrow\infty} \left(1 + \frac{r}{A - r - 1}\right) \\
    =& \ln e^{2r} + \left(2r + 1 - h - w\right) \ln 1\\
    =& 2r
\end{align*}

이고, 따라서 $hw$의 미로에서 벽을 $r$개 지운다면
평균적으로 $2r$번의 연산이 필요하다. 결론적으로 \underline{평균적인 경우} 완전 미로에서 벽들을 더 지워 불완전 미로를 만드는 데 드는 시간 복잡도는
$\mathcal{O}\left(r\right)$으로 계산 가능하다. 불완전 미로를 처음부터 새로 만드는 경우를 생각한다면
완전 미로를 먼저 만들어야 하므로 전체 시간 복잡도는 $\mathcal{O}\left(h\times w + r\right)$이다.

자료구조는 동일한 것을 사용하였으므로 공간 복잡도도 마찬가지로 $\mathcal{O}\left(h\times w\right)$이다.