구현 1주차에는 완전 미로와 불완전 미로 생성 로직을 각각 구현한다. 각각의 미로 생성에 필요한 자료구조와 알고리즘을 고안한다.

구현 2주차에는 1주차에서 생성한 미로를 읽어 화면에 그리는 GUI를 구성한다. 이를 위해 다음과 같은 항목들의 구현이 필요하다.
\begin{itemize}
    \item 화면에 UI 요소를 그리는 로직
    \item 미로를 표현하기 위한 자료구조의 고안
\end{itemize}

구현 3주차에는 미로 해결 알고리즘을 구현한다. 이를 위해 다음 알고리즘의 이해와 구현이 필요하다.
\begin{itemize}
    \item 깊이 우선 탐색\translation{DFS}
    \item 너비 우선 탐색\translation{BFS}
\end{itemize}