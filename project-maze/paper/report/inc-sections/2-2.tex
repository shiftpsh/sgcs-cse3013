\subsubsection{깊이 우선 탐색\translation{depth first search}} 깊이 우선 탐색, 또는 DFS는 그래프 탐색 방법의 일종이다.

$N\left(u\right)$를 $u$와 인접한 노드들의 집합이라고 정의한다면, $N\left(u\right)$의
모든 원소 $v_i$에 대해 $N\left(v_i\right)$를 전부 방문한 후 $v_{i + 1}$을 방문하는 방식으로 동작한다.
이를 위해 스택을 사용한다.

간단히 말하자면, DFS는 진행 방향을 정하고 갈 수 있는 만큼 진행한 후 더 이상 진행할 수 없으면 (막다른 길을 마주하면)
이전 노드로 되돌아와 다른 경로로 진행하는 방식으로 동작한다. 이 때 이전 노드로 되돌아오는 것을 백트래킹\translation{backtracking}이라 한다.
시작점부터 종점까지의 경로를 성공적으로 찾았을 경우 DFS의 스택에는 그 경로가 저장되어 있다는 특징이 있다.

DFS의 시간 복잡도는 그래프 $\left(V,\,E\right)$에 대해 인접 행렬로 표현된 그래프의 경우 $\mathcal{O}\left(V^2\right)$,
인접 리스트로 표현된 그래프의 경우 $\mathcal{O}\left(V+E\right)$를 보인다.

\subsubsection{너비 우선 탐색\translation{breadth first search}} 너비 우선 탐색, 또는 BFS는 DFS와 마찬가지로 그래프 탐색 방법의 일종이다.

BFS는 큐 $Q$에 첫 노드를 넣고, $Q$가 빌 때까지 $Q$의 첫 원소 $q$에 대해 $N\left(q\right)$ 중 아직 방문하지 않은 정점들을 전부 $Q$에 추가하면서 그래프를 탐색해 나간다.
다시 말하면 인접한 노드들부터 차례로 탐색해나가며, 트리의 경우 깊이가 얕은 노드부터 탐색해나간다. 이는 BFS가 시작점부터의 거리가 $n+1$인 정점을 방문했을 경우
시작점으로부터의 거리가 $n$인 정점은 이미 모두 방문했음을 의미하며, 간선의 가중치가 모두 같은 경우 이는 최단 경로를 찾는 데 쓰일 수 있다.

BFS의 시간 복잡도는 그래프 $\left(V,\,E\right)$에 대해 인접 행렬로 표현된 그래프의 경우 마찬가지로 $\mathcal{O}\left(V^2\right)$,
인접 리스트로 표현된 그래프의 경우 $\mathcal{O}\left(V+E\right)$를 보인다.

\subsubsection{재귀 백트래킹\translation{recursive backtracker}} 재귀 백트래킹은 완전 미로를 만드는 알고리즘 중 하나이다.
완전 미로는 최소 스패닝 트리\translation{minimum spanning tree}로 생각할 수 있는데, 이 점에서 착안해 미로의 모든 칸을 노드로 생각한다면
랜덤한 점에서 랜덤하게 MST를 만드는 식으로 완전 미로를 구성할 수 있다. 따라서 DFS를 응용해 MST를 구성하여 미로를 생성할 수 있다.