\documentclass[runningheads]{../../../llncs}
\usepackage[paperheight=295mm,paperwidth=210mm]{geometry}
\usepackage{graphicx}
\usepackage{import}
\usepackage{kotex}
\usepackage[dvipsnames]{xcolor}
\usepackage{fancyvrb}
\usepackage{listings}
\usepackage{indentfirst}
\usepackage{tabularx}
\usepackage{underscore}
\usepackage{multicol}
\usepackage{enumitem}
\usepackage{menukeys}
\usepackage{amsmath}
\usepackage{clrscode3e} % https://www.ctan.org/pkg/clrscode3e?lang=en
\usepackage[numbers,square,super]{natbib}
\usepackage{inconsolata} % Inconsolata
\usepackage{mathptmx} % Times New Roman
\usepackage{minted}
\graphicspath{ {./images/} }
\lstset{basicstyle=\footnotesize\ttfamily,breaklines=true}
\renewcommand{\bibname}{참고문헌}
\setlength{\parindent}{1em}
\setlength{\parskip}{1em}
\linespread{1.2}
{\renewcommand{\arraystretch}{1.5}%
\setlength{\tabcolsep}{0.5em}%
\newenvironment{Figure}
  {\par\medskip\noindent\minipage{\linewidth}}
  {\endminipage\par\medskip}
\newcommand{\translation}[1]{\textsuperscript{#1}}
\newlist{algorithm}{enumerate}{10}
\setlist[algorithm]{label*=\arabic*.}
\setlist[algorithm,1]{label=\textbf{\arabic*}}
\setlist[algorithm,2]{label=\textbf{\alph*}}
\setlist[algorithm,3]{label=\textbf{\roman*}}
\setlist[algorithm,4]{label=(\arabic*)}
\setlist[algorithm,5]{label=(\alph*)}
\setlist[algorithm,6]{label=(\roman*)}
\makeatletter
\renewcommand\NAT@citesuper[3]{\ifNAT@swa
\if*#2*\else#2\NAT@spacechar\fi
\unskip\kern\p@\textsuperscript{\NAT@@open#1\if*#3*\else,\NAT@spacechar#3\fi\NAT@@close}%
   \else #1\fi\endgroup}
\makeatother

\usepackage{color}
\usepackage{pgfplots}

\newdimen\omsq     \omsq=20pt
\newdimen\omrule   \omrule=2pt
\newdimen\omint

\newif\ifvth    \newif\ifhth    \newif\ifomblank
\def\OMINO#1{%
    \vthtrue \hthtrue
    \vbox{ \offinterlineskip\parindent=0pt \OM#1\relax\vskip1pt}
    }

\def\OM#1{%
    \omint=\omsq    \advance\omint-\omrule
    \ifx\relax#1%
    \else
      \ifx\\#1 \newline\null \hthtrue \ifvth\vthfalse\else\vskip-\omrule\vthtrue\fi
      \else%
        \ifx .#1\hskip\ifhth \omrule\else \omint\fi
        \else%
          \ifx +#1\def\colour{black}\fi%
          \ifx -#1\def\colour{black}\fi%
          \ifx |#1\def\colour{black}\fi%
          \ifx @#1\def\colour{black}\fi%
          \ifx X#1\def\colour{gray}\fi%
          \ifx Z#1\def\colour{red}\fi%
          \ifx S#1\def\colour{green}\fi%
          \ifx L#1\def\colour{orange}\fi%
          \ifx J#1\def\colour{blue}\fi%
          \ifx O#1\def\colour{yellow}\fi%
          \ifx T#1\def\colour{magenta}\fi%
          \ifx I#1\def\colour{cyan}\fi%
          \textcolor{\colour}{\rule{\ifhth\omrule\else\omsq\fi}{\ifvth\omrule\else\omsq\fi}}%
          \ifhth\else\hskip -\omrule\fi%
        \fi%
        \ifhth\hthfalse\else\hthtrue\fi%
      \fi%
    \expandafter\OM%
    \fi}

\makeatother

\usepackage{wrapfig}
\usepackage{subfig}

\begin{document}

\title{CSE3013 (컴퓨터공학 설계 및 실험 I) \space \newline 미로 프로젝트}
\author{서강대학교 컴퓨터공학과 박수현 (20181634)}
\institute{제 2분반; 담당교수: 서강대학교 컴퓨터공학과 장형수}
\maketitle

\section{설계 문제 및 목표}

미로 프로젝트에서는 미로를 만드는 프로그램과 미로를 보여주는 GUI 프로그램, 그리고 미로에서 가장 짧은 경로를 찾는 프로그램을 제작한다.

미로 프로젝트 1주차 실습에선 완전 미로를 생성하는 프로그램을 제작한다. 완전 미로란 임의의 서로 다른 출발점과 도착점을 연결하는 경로가 오직 하나만 존재하는 미로를 의미한다.
완전미로를 생성하기 위한 알고리즘으로 recursive backtracker, Kruskal's algorithm, Prim's algorithm, Eller's algorithm 중 하나를 택할 수 있다.
1주차 숙제는 순환 경로가 존재하는 불완전한 미로를 생성하는 프로그램을 제작하는 것이다. 순환 경로가 존재하는 불완전한 미로란 폐쇄된 공간은 존재하지 않으나 두 지점을 연결하는 경로가
하나 이상 존재하는 미로를 말한다. 실습과 숙제 모두 확장자가 .maz인 파일로 미로를 출력한다.

미로 프로젝트 2주차 실습에선 1주차에서 만든 프로그램이 생성한 미로(확장자가 .maz인 파일의 내용)를 읽어 들여, MFC가 제공하는 윈도우 환경에서의 GUI로 그 미로를 그려내는 프로그램을
제작해야 한다. 이를 위해 미로를 표현하기 위한 효율적인 자료구조를 선택하여 활용한다. 여기에 3주차 프로젝트 내용을 위한 DFS 버튼과 BFS 버튼을 추가하는 것이 2주차의 숙제이다.

미로 프로젝트 3주차에선 2주차에서 만들어진 프로그램에 가장 짧은 경로를 찾는 기능을 제공하도록 하는 것이다. 경로를 찾는 방법은 DFS(3주차 실습)와 BFS(3주차 숙제)를 사용한다.
이 때, MFC의 GUI를 이용하여 탐색 과정과 결과를 사용자에게 보여주어야 한다.

\section{요구사항}
\subsection{설계 목표 설정}
% 미로 프로젝트의 각 주차별 프로그램을 설계하는 과정에서의 목표를 설정하고, 그 과정에서 요구되는 사항들을 정리한다.
\import{inc-sections/}{2-1.tex}

\subsection{합성}
% 미로 프로젝트의 각 주차별 프로그램의 설계 및 구현을 위해 요구되는 이론, 자료구조, 알고리즘 등을 조사 분석하여 전체적인 설계를 수행한다.
\import{inc-sections/}{2-2.tex}

\newpage
\subsection{분석}
% 미로 프로젝트의 각 주차별 문제에 대해 기술하고, 상기 단계에서 설계한 알고리즘에 기반을 두어 목표로 하는 프로그램을 효과적으로 개발하는데 필요한 프로그램 기법들과
% 자료구조에 대하여 조사 분석하여 그것을 바탕으로 전체 프로그램 순서도를 작성한다. 또한 이러한 프로그램을 개발하는데 있어 고려해야할 모든 요소, 예를 들어, 입출력 양식,
% 관련 자료구조와 이론, 사용할 C언어 함수의 사용법 등 모든 가능한 고려 사항을 정리한다.
\import{inc-sections/}{2-3-week1.tex}
\import{inc-sections/}{2-3-week2.tex}
\import{inc-sections/}{2-3-week3.tex}

\subsection{제작}
% 위에서 설계한 내용을 C언어를 사용해 구현한다. 구현 후 프로그램의 각 구성요소를 상세히 분석하여 구현방법을 프로그램 기법과 정리한 이론 등과 연관 지어 정리한다.

% For some reasons, my minted doesn't recognize files with its filename containing '_'.
\import{inc-sections/}{2-4.tex}

\subsection{시험}
% 위의 과정에서 수행한 문제 정의, 프로그램 순서도와 순서도 상의 각 부분 역할 및 구현 방법, 프로그램의 구현 방법, 구현 방법의 이론과의 연관성, 구현한 프로그램의 내용, 수행화면 등을 정리한다. 
\import{inc-sections/}{2-5.tex}

\subsection{평가}
% 문제와 이론의 연관성이 적절한지, 순서도와 실제 구현 사이에 차이는 없는지, 프로그램은 잘 동작하는지 등을 평가한다. 
\import{inc-sections/}{2-6.tex}

\subsection{환경}
MFC를 이용한 윈도우 프로그래밍이 요구되므로, 미로 프로젝트의 각 프로그램은 윈도우 환경에서 제작된다.
% (이 내용은 그대로 둘 것)

\subsection{미학}
프로젝트에서 정의하고 있는 미로의 모습대로 미로를 만들어야 하고, 이를 GUI 프로그램으로 그려내는 경우 또한 프로젝트에서 정의하고 있는 모습대로 그려야 한다. 

\subsection{보건 및 안정}
% 프로젝트에서 정의하고 있지 않은 입력으로 인한 프로그램의 오동작이 없어야 한다. 또한 프로그램 동작 중에 일어날 수 있는 오류에 대처할 수 있어야 한다.
입력되는 미로 파일의 포맷이 잘못되었을 경우 오동작이 발생할 수 있다. 이외의 경우 프로그램은 안정적이다.

\section{기타}
\subsection{환경 구성}
% 실제 프로젝트를 수행한 환경에 대해 구체적으로 기술한다. 프로젝트 수행에 사용된 하드웨어 및 소프트웨어의 상세 정보를 정리한다.
\import{inc-sections/}{3-1.tex}

\subsection{참고 사항}
% 프로젝트의 수행 시 참고 사항에 대해서 기술한다.(참고 사항이 없으면 ‘참고사항 없음’, 있다면 그 내용을 적는다.)
특별히 참고할 만한 사항은 없다.

\subsection{팀 구성}
% (팀 구성 관련 사항을 기입. 예: ‘첫 번째 프로젝트 팀 구성을 유지한다. 본 프로젝트에 대한 각 구성원의 기여사항을 보고서에 명시하도록 한다.’
% 이 프로젝트는 개인 프로젝트이므로 팀 구성이 1명이다. 자신의 이름을 쓰고 100%로 기술하도록 한다.)
서강대학교 컴퓨터공학과 박수현 (20181634) 100\% 기여. 개인 프로젝트로 진행하였다.

\subsection{수행기간}
% (테트리스 1주차 실습 수업일부터 제출 마감일까지로 기입)
2019년 5월 27일부터 2019년 6월 21일까지.

\end{document}
