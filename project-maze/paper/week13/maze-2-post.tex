\documentclass[runningheads]{../../../llncs}
\usepackage[paperheight=295mm,paperwidth=210mm]{geometry}
\usepackage{graphicx}
\usepackage{import}
\usepackage{kotex}
\usepackage[dvipsnames]{xcolor}
\usepackage{fancyvrb}
\usepackage{listings}
\usepackage{indentfirst}
\usepackage{tabularx}
\usepackage{underscore}
\usepackage{multicol}
\usepackage{enumitem}
\usepackage{menukeys}
\usepackage{amsmath}
\usepackage{clrscode3e} % https://www.ctan.org/pkg/clrscode3e?lang=en
\usepackage[numbers,square,super]{natbib}
\usepackage{inconsolata} % Inconsolata
\usepackage{mathptmx} % Times New Roman
\usepackage{minted}
\graphicspath{ {./images/} }
\lstset{basicstyle=\footnotesize\ttfamily,breaklines=true}
\renewcommand{\bibname}{참고문헌}
\setlength{\parindent}{1em}
\setlength{\parskip}{1em}
\linespread{1.2}
{\renewcommand{\arraystretch}{1.5}%
\setlength{\tabcolsep}{0.5em}%
\newenvironment{Figure}
  {\par\medskip\noindent\minipage{\linewidth}}
  {\endminipage\par\medskip}
\newcommand{\translation}[1]{\textsuperscript{#1}}
\newlist{algorithm}{enumerate}{10}
\setlist[algorithm]{label*=\arabic*.}
\setlist[algorithm,1]{label=\textbf{\arabic*}}
\setlist[algorithm,2]{label=\textbf{\alph*}}
\setlist[algorithm,3]{label=\textbf{\roman*}}
\setlist[algorithm,4]{label=(\arabic*)}
\setlist[algorithm,5]{label=(\alph*)}
\setlist[algorithm,6]{label=(\roman*)}
\makeatletter
\renewcommand\NAT@citesuper[3]{\ifNAT@swa
\if*#2*\else#2\NAT@spacechar\fi
\unskip\kern\p@\textsuperscript{\NAT@@open#1\if*#3*\else,\NAT@spacechar#3\fi\NAT@@close}%
   \else #1\fi\endgroup}
\makeatother
\newcommand{\norm}[1]{\left\lVert#1\right\rVert}

\begin{document}

\title{CSE3013 (컴퓨터공학 설계 및 실험 I) \space \newline PRJ-2 미로 프로젝트 2주차 결과 보고서}
\author{서강대학교 컴퓨터공학과 박수현 (20181634)}
\institute{서강대학교 컴퓨터공학과}
\maketitle

\section{목적}
실습에서 구성한 자료 구조를 기술한다.

\section{문제}
실습에서는 자료 구조의 구현으로 \mintinline{c++}{std::vector<std::vector<char>>}를 사용하였다. 빈 칸은 `.', 벽은 `+', `-', `$|$' 중 하나로
이루어진 2차원 배열이다. .maz 파일을 읽어들여 그대로 2차원 배열에 저장하는 방식이다.

자료 구조의 내용을 스크린에 그리는 코드는 다음과 같다.
\begin{minted}[xleftmargin=\parindent,breaklines,linenos]{c++}
for (int i = 0; i < rows; i++) {
    int x = CELL_SIZE * (i / 2) + WALL_WIDTH * (i - i / 2);
    for (int j = 0; j < cols; j++) {
        int y = CELL_SIZE * (j / 2) + WALL_WIDTH * (j - j / 2);

        // this adds gradients just for fun
        double blend = (double) (i + j) / (rows + cols);
        COLORREF color = RGB(
            0x00 * (1 - blend) + 0x9c * blend,
            0xbc * (1 - blend) + 0x27 * blend,
            0xd4 * (1 - blend) + 0xb0 * blend
            );

        switch (field[i][j]) {
            case '+':
                DrawSolidBox_I(y, x,
                    y + WALL_WIDTH, x + WALL_WIDTH,
                    0, color, color);
                break;
            case '-':
                DrawSolidBox_I(y, x,
                    y + CELL_SIZE, x + WALL_WIDTH,
                    0, color, color);
                break;
            case '|':
                DrawSolidBox_I(y, x,
                    y + WALL_WIDTH, x + CELL_SIZE,
                    0, color, color);
                break;
        }
    }
}
\end{minted}

이는 직사각형을 $rows \times cols$번 그리는데, 지금 확인하는 칸이 `+'일 경우 작은 정사각형을, `-'일 경우 좌우로 긴 직사각형을, `$|$'일 경우 상하로
긴 직사각형을 적절한 위치에 그린다. 크기는 \texttt{WALL_WIDTH}와 \texttt{CELL_SIZE}에 정의되어 있다.

$i$와 $j$가 홀수일 경우는 현재 확인하고 있는 칸은 원래 미로의 각 칸이 되는 경우이고 $i$ 혹은 $j$가 짝수일 경우는 원래 미로의 벽 혹은 통로가 되는 경우이다.
따라서 이 점을 이용해 직사각형을 그리기 시작하는 왼쪽 위 $x$ 좌표는 $i$에 대해
\[\left(CELL\_SIZE \times \left\lfloor\frac{i}{2}\right\rfloor\right) + \left(WALL\_WIDTH \times \left\lceil\frac{i}{2}\right\rceil\right)\]
이고, 이는 $y$ 좌표에 대해서도 마찬가지이다. 직사각형을 그리는 작업이 상수 시간만큼 걸린다고 가정하면 미로를 그리는 데 소요되는 시간 복잡도는 $\mathcal{O}\left(rows \times cols\right)$
이고 미로를 저장하는 공간 복잡도도 $\mathcal{O}\left(rows \times cols\right)$이다.

실험 전에 생각한 방법과 다르게 이렇게 구현하더라도 `.' 칸들에만 정점이, 인접한 `.' 칸들 사이에만 간선이 있다고 생각하면
미로를 그래프처럼 생각하고 탐색할 수 있으므로 굳이 가로/세로 길을 계산해 저장할 필요가 없다.

\section{습득한 내용}
MFC에서 리소스를 수정하여 메뉴 항목을 추가하거나 툴바 아이콘을 추가하는 방법을 알 수 있었다.

\end{document}
