\documentclass[runningheads]{../../../llncs}
\usepackage[paperheight=295mm,paperwidth=210mm]{geometry}
\usepackage{graphicx}
\usepackage{import}
\usepackage{kotex}
\usepackage[dvipsnames]{xcolor}
\usepackage{fancyvrb}
\usepackage{listings}
\usepackage{indentfirst}
\usepackage{tabularx}
\usepackage{underscore}
\usepackage{multicol}
\usepackage{enumitem}
\usepackage{menukeys}
\usepackage{amsmath}
\usepackage{clrscode3e} % https://www.ctan.org/pkg/clrscode3e?lang=en
\usepackage[numbers,square,super]{natbib}
\usepackage{inconsolata} % Inconsolata
\usepackage{mathptmx} % Times New Roman
\usepackage{minted}
\graphicspath{ {./images/} }
\lstset{basicstyle=\footnotesize\ttfamily,breaklines=true}
\renewcommand{\bibname}{참고문헌}
\setlength{\parindent}{1em}
\setlength{\parskip}{1em}
\linespread{1.2}
{\renewcommand{\arraystretch}{1.5}%
\setlength{\tabcolsep}{0.5em}%
\newenvironment{Figure}
  {\par\medskip\noindent\minipage{\linewidth}}
  {\endminipage\par\medskip}
\newcommand{\translation}[1]{\textsuperscript{#1}}
\newlist{algorithm}{enumerate}{10}
\setlist[algorithm]{label*=\arabic*.}
\setlist[algorithm,1]{label=\textbf{\arabic*}}
\setlist[algorithm,2]{label=\textbf{\alph*}}
\setlist[algorithm,3]{label=\textbf{\roman*}}
\setlist[algorithm,4]{label=(\arabic*)}
\setlist[algorithm,5]{label=(\alph*)}
\setlist[algorithm,6]{label=(\roman*)}
\makeatletter
\renewcommand\NAT@citesuper[3]{\ifNAT@swa
\if*#2*\else#2\NAT@spacechar\fi
\unskip\kern\p@\textsuperscript{\NAT@@open#1\if*#3*\else,\NAT@spacechar#3\fi\NAT@@close}%
   \else #1\fi\endgroup}
\makeatother
\newcommand{\norm}[1]{\left\lVert#1\right\rVert}

\begin{document}

\title{CSE3013 (컴퓨터공학 설계 및 실험 I) \space \newline PRJ-2 미로 프로젝트 2주차 예비 보고서}
\author{서강대학교 컴퓨터공학과 박수현 (20181634)}
\institute{서강대학교 컴퓨터공학과}
\maketitle

\section{목적}
DFS와 BFS 알고리즘을 이해하고 미로 문제 해결에 어떻게 사용할 수 있을 것인지 보인다.

\section{문제}
\subsection{그래프 탐색 알고리즘}

너비 우선 탐색\translation{Breadth-First Search}과 깊이 우선 탐색\translation{Depth-First Search}는 그래프를
탐색하는 대표적인 알고리즘들이다.

어떤 노드 $u$와 인접한 노드들의 집합을 $N\left(u\right)$라고 하자. BFS는 큐 $Q$에 첫 노드를 넣고,
$Q$가 빌 때까지 $Q$의 첫 원소 $q$에 대해 $N\left(q\right)$ 중 아직 방문하지 않은 정점들을 전부 $Q$에 추가하면서 그래프를 탐색해 나간다.
다시 말하면 인접한 노드들부터 차례로 탐색해나가며, 트리의 경우 깊이가 얕은 노드부터 탐색해나간다.

DFS는 BFS와 반대의 전략을 취하는데, $N\left(u\right)$의 모든 원소 $v_i$에 대해 $N\left(v_i\right)$를 전부 방문한 후 $v_{i + 1}$을
방문하는 방식으로 동작한다. 이를 위해 스택을 사용한다. DFS는 한 방향으로 깊이 탐색하다가 더 이상 진행할 수 없으면 이전 노드로 되돌아와 다른
방향으로 탐색하는데, 이전 노드로 되돌아오는 것을 백트래킹\translation{backtracking}이라 한다.

인접 리스트로 표현된 그래프 $G = \left(V,\,E\right)$의 경우 각각의 방법의 시간 복잡도는 모두 $\mathcal{O}\left(\norm{V}+\norm{E}\right)$
를 보인다.

\subsection{미로 문제에의 적용}

$h\times w$ 미로의 $\left(i,\,j\right)$에 위치한 셀을 $u_{ij}$라 하자. 그러면 $u$는 그래프의 노드로 생각할 수 있으며,
이웃한 노드는 최대 4개 -- $u_{\left(i - 1\right)j}$, $u_{\left(i + 1\right)j}$,
$u_{i\left(j - 1\right)}$, $u_{i\left(j + 1\right)}$ -- 가 가능하다. 이는 미로의 가장자리에 있는 노드들을 제외하고
모두 동일하므로 간선 정보는 1주차에사 사용한 자료구조를 그대로 사용하고, 추가로 DFS/BFS의 수행을 위해 방문 순서와 여부를 저장하는 $h\times w$의
배열을 도입한다. 이 때 공간 복잡도는 $\mathcal{O}\left(h\times w\right)$가 된다.
또한 이 경우 두 알고리즘 공통적으로 현재 노드에서 인접한 4방향에 존재하는 노드 중 벽으로 막혀 있지 않으며
아직 방문하지 않은 노드를 쿼리하는 것은 단순히 2차원 배열의 원소에 4번 접근하는 것과 동일하다.

\subsection{탐색 알고리즘의 구현 방법}

\begin{itemize}
    \item \textbf{DFS} -- 인접한 노드를 방문하는 재귀함수 \proc{DFS}($u$)를 정의한다. $u$에 대해
    $v \in N\left(u\right)$마다, $v$를 아직 방문하지 않았다면 \proc{DFS}($v$)를 호출한다.
    \item \textbf{BFS} -- 큐 $Q$를 정의한다. 위에서 소개한 대로 $Q$에 첫 노드를 넣고,
    $Q$의 첫 원소 $q$에 대해 $N\left(q\right)$ 중 아직 방문하지 않은 정점들을 전부 $Q$에 추가하고,
    $Q$의 첫 원소를 제거한다. 이를 $Q$가 빌 때까지 반복한다. 
\end{itemize}

\end{document}
