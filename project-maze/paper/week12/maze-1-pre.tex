\documentclass[runningheads]{../../../llncs}
\usepackage[paperheight=295mm,paperwidth=210mm]{geometry}
\usepackage{graphicx}
\usepackage{import}
\usepackage{kotex}
\usepackage[dvipsnames]{xcolor}
\usepackage{fancyvrb}
\usepackage{listings}
\usepackage{indentfirst}
\usepackage{tabularx}
\usepackage{underscore}
\usepackage{multicol}
\usepackage{enumitem}
\usepackage{menukeys}
\usepackage{amsmath}
\usepackage{clrscode3e} % https://www.ctan.org/pkg/clrscode3e?lang=en
\usepackage[numbers,square,super]{natbib}
\usepackage{inconsolata} % Inconsolata
\usepackage{mathptmx} % Times New Roman
\usepackage{minted}
\graphicspath{ {./images/} }
\lstset{basicstyle=\footnotesize\ttfamily,breaklines=true}
\renewcommand{\bibname}{참고문헌}
\setlength{\parindent}{1em}
\setlength{\parskip}{1em}
\linespread{1.2}
{\renewcommand{\arraystretch}{1.5}%
\setlength{\tabcolsep}{0.5em}%
\newenvironment{Figure}
  {\par\medskip\noindent\minipage{\linewidth}}
  {\endminipage\par\medskip}
\newcommand{\translation}[1]{\textsuperscript{#1}}
\newlist{algorithm}{enumerate}{10}
\setlist[algorithm]{label*=\arabic*.}
\setlist[algorithm,1]{label=\textbf{\arabic*}}
\setlist[algorithm,2]{label=\textbf{\alph*}}
\setlist[algorithm,3]{label=\textbf{\roman*}}
\setlist[algorithm,4]{label=(\arabic*)}
\setlist[algorithm,5]{label=(\alph*)}
\setlist[algorithm,6]{label=(\roman*)}
\makeatletter
\renewcommand\NAT@citesuper[3]{\ifNAT@swa
\if*#2*\else#2\NAT@spacechar\fi
\unskip\kern\p@\textsuperscript{\NAT@@open#1\if*#3*\else,\NAT@spacechar#3\fi\NAT@@close}%
   \else #1\fi\endgroup}
\makeatother
	
\begin{document}

\title{CSE3013 (컴퓨터공학 설계 및 실험 I) \space \newline PRJ-2 미로 프로젝트 1주차 예비 보고서}
\author{서강대학교 컴퓨터공학과 박수현 (20181634)}
\institute{서강대학교 컴퓨터공학과}
\maketitle

\section{목적}
마로 게임을 위한 알고리즘과 자료구조를 이해한다.

\section{문제}
\subsection{미로 생성 알고리즘}

미로의 각 칸들을 정점이라고 생각하고, 인접 관계를 간선이라고 생각한다면, 미로는 그래프로 나타낼 수 있다, 여기서 완전 미로는
최소 비용 신장 트리\translation{minimum spanning tree}로 생각할 수 있다. 그래프에서 최소 비용 신장 트리를 만드는
알고리즘 중 하나로 프림 알고리즘\translation{Prim's algorithm}이 있다.
프림 알고리즘은 최단 경로 탐색 알고리즘인 데이크스트라 알고리즘\translation{Djikstra's algorithm}과 비슷한
방식으로 동작힌디.

임의의 노드를 골라 트리의 루트로 만들고, 그래프의 모든 변이 들어 있는 최소 힙 $Q$에 대해,
모든 노드가 트리에 포함되어 있지 않은 동안, 트리와 연결된 간선 가운데 트리 속의 두 꼭짓점을 연결하지 않는 가장 작은 간선을 트리에 추가한다.
이를 의사 코드로 나타내면 다음과 같다.

\begin{codebox}
\Procname{\proc{MST-Prim}($G$, $w$, $r$)}
\li \For each $u \in G.V$ \Do
\li     $\attrib{u}{key} \gets \infty$
\li     $\attrib{u}{\pi} \gets \const{null}$
    \End
\li $\attrib{r}{key} \gets 0$
\li $Q \gets \attrib{G}{V}$
\li \While $Q \neq \emptyset$ \Do
\li     $u \gets \proc{Extract-Min}(Q)$
\li     \For each $v \in \attrib{G}{Adj}[u]$ \Do
\li         \If $v \in Q$ and $w(u,\,v) < \attrib{v}{key}$ \Then
\li             $\attrib{v}{\pi} \gets u$
\li             $\attrib{v}{key} \gets w(u,\,v)$
            \End
        \End
    \End
\end{codebox}

하지만 이 프로젝트의 경우 간선들의 길이는 항상 1이므로, 최소 힙 등을 신경쓰지 않고도 알고리즘을 진행할 수 있겠다. 
모든 노드가 트리에 포함되어 있지 않은 동안, 남아 있는 간선 중 트리와 연결된 간선 가운데 아무 간선이나 랜덤하게 골라서 추가하면 되는 것이다.

$n\times m$의 격자를 그래프로 만들면 노드는 $n\times m$개, 간선은 $m\left(n - 1\right) + n\left(m - 1\right)$개이다.
모든 간선마다 한 꼭짓점은 트리에 포함되어 있고 다른 꼭지점은 아닌 간선을 하나 골라 트리에 추가하기만 하면 되므로 간선 개수만큼의 시간이 소요된다.

이렇게 얻은 최소 비용 신장 트리를 인접 리스트 자료구조로 저장한다고 가정하자. 트리의 노드는 $nm$개이므로 간선은 $nm - 1$개이다. 따라서 
이 자료구조의 공간 복잡도는 $\mathcal{O}\left(nm\right)$이 된다.

\end{document}
