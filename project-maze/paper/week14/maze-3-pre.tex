\documentclass[runningheads]{../../../llncs}
\usepackage[paperheight=295mm,paperwidth=210mm]{geometry}
\usepackage{graphicx}
\usepackage{import}
\usepackage{kotex}
\usepackage[dvipsnames]{xcolor}
\usepackage{fancyvrb}
\usepackage{listings}
\usepackage{indentfirst}
\usepackage{tabularx}
\usepackage{underscore}
\usepackage{multicol}
\usepackage{enumitem}
\usepackage{menukeys}
\usepackage{amsmath}
\usepackage{clrscode3e} % https://www.ctan.org/pkg/clrscode3e?lang=en
\usepackage[numbers,square,super]{natbib}
\usepackage{inconsolata} % Inconsolata
\usepackage{mathptmx} % Times New Roman
\usepackage{minted}
\graphicspath{ {./images/} }
\lstset{basicstyle=\footnotesize\ttfamily,breaklines=true}
\renewcommand{\bibname}{참고문헌}
\setlength{\parindent}{1em}
\setlength{\parskip}{1em}
\linespread{1.2}
{\renewcommand{\arraystretch}{1.5}%
\setlength{\tabcolsep}{0.5em}%
\newenvironment{Figure}
  {\par\medskip\noindent\minipage{\linewidth}}
  {\endminipage\par\medskip}
\newcommand{\translation}[1]{\textsuperscript{#1}}
\newlist{algorithm}{enumerate}{10}
\setlist[algorithm]{label*=\arabic*.}
\setlist[algorithm,1]{label=\textbf{\arabic*}}
\setlist[algorithm,2]{label=\textbf{\alph*}}
\setlist[algorithm,3]{label=\textbf{\roman*}}
\setlist[algorithm,4]{label=(\arabic*)}
\setlist[algorithm,5]{label=(\alph*)}
\setlist[algorithm,6]{label=(\roman*)}
\makeatletter
\renewcommand\NAT@citesuper[3]{\ifNAT@swa
\if*#2*\else#2\NAT@spacechar\fi
\unskip\kern\p@\textsuperscript{\NAT@@open#1\if*#3*\else,\NAT@spacechar#3\fi\NAT@@close}%
   \else #1\fi\endgroup}
\makeatother
\newcommand{\norm}[1]{\left\lVert#1\right\rVert}

\begin{document}

\title{CSE3013 (컴퓨터공학 설계 및 실험 I) \space \newline PRJ-2 미로 프로젝트 3주차 예비 보고서}
\author{서강대학교 컴퓨터공학과 박수현 (20181634)}
\institute{서강대학교 컴퓨터공학과}
\maketitle

\section{목적}
DFS와 BFS 알고리즘의 시간 복잡도를 보이고 설계한 자료구조에서 각각의 알고리즘을 이용해 어떻게 경로를 찾을 수 있는지 보인다.

\section{문제}
\subsection{그래프 탐색 알고리즘의 시간 복잡도}

그래프 $G = \left(V,\,E\right)$가 있다고 하자. DFS와 BFS는 모두 모든 정점을 한 번씩만 방문하는 알고리즘이므로 정점을 방문하는 데에는
$\mathcal{O}\left(\norm{V}\right)$가 소요된다. 또한 정점에 대해 인접한 정점을 쿼리해 진행해야 하는데, 인접한 정점을 쿼리하는 연산은
어느 자료 구조를 사용하느냐에 따라 시간 복잡도가 달라진다. 인접 리스트를 사용할 경우 인접한 정점을 쿼리하는 연산의 시간 복잡도는
$\mathcal{O}\left(\norm{E}\right)$이고, 인접 행렬을 사용할 경우 각 노드마다 연결 여부를 체크해야 하므로 전체 시간 복잡도는
$\mathcal{O}\left(\norm{V}^2\right)$이다.

따라서 전체 시간 복잡도는 인접 리스트의 경우 $\mathcal{O}\left(\norm{V}+\norm{E}\right)$이고 인접 행렬의 경우
$\mathcal{O}\left(\norm{V}^2\right)$이다.

\subsection{탐색 알고리즘의 구현 방법}

\begin{itemize}
    \item \textbf{DFS} -- 스택 $S$를 정의한다. $S$에 첫 노드를 넣고,
    $S$의 첫 원소 $s$에 대해 $N\left(s\right)$ 중 아직 방문하지 않은 정점들을 전부 $S$에 추가하고,
    $S$의 첫 원소를 제거한다. 이를 $S$가 빌 때까지 반복한다. 
    \item \textbf{BFS} -- 큐 $Q$를 정의한다. $Q$에 첫 노드를 넣고,
    $Q$의 첫 원소 $q$에 대해 $N\left(q\right)$ 중 아직 방문하지 않은 정점들을 전부 $Q$에 추가하고,
    $Q$의 첫 원소를 제거한다. 이를 $Q$가 빌 때까지 반복한다. 
\end{itemize}

구현한 자료구조의 크기가 $col \times row$라 히면, 가장자리는 모두 벽이므로 맨 왼쪽 위 칸을 $\left(0,\,0\right)$이라 할 때
첫 노드는 $\left(1,\,1\right)$이고 마지막 노드는 $\left(col - 2,\,row - 2\right)$이다. 인접한 노드는 인접한 4방향의
칸 중 통로(`.' 문자) 칸으로 정의한다.

\end{document}
