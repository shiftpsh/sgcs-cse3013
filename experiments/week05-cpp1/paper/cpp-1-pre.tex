\documentclass[runningheads]{../../../llncs}
\usepackage[paperheight=295mm,paperwidth=210mm]{geometry}
\usepackage{graphicx}
\usepackage{import}
\usepackage{kotex}
\usepackage[dvipsnames]{xcolor}
\usepackage{fancyvrb}
\usepackage{listings}
\usepackage{indentfirst}
\usepackage{tabularx}
\usepackage{underscore}
\usepackage{multicol}
\usepackage{enumitem}
\usepackage{menukeys}
\usepackage{amsmath}
\usepackage{clrscode3e} % https://www.ctan.org/pkg/clrscode3e?lang=en
\usepackage[numbers,square,super]{natbib}
\usepackage{inconsolata} % Inconsolata
\usepackage{mathptmx} % Times New Roman
\usepackage{minted}
\graphicspath{ {./images/} }
\lstset{basicstyle=\footnotesize\ttfamily,breaklines=true}
\renewcommand{\bibname}{참고문헌}
\setlength{\parindent}{1em}
\setlength{\parskip}{1em}
\linespread{1.2}
{\renewcommand{\arraystretch}{1.5}%
\setlength{\tabcolsep}{0.5em}%
\newenvironment{Figure}
  {\par\medskip\noindent\minipage{\linewidth}}
  {\endminipage\par\medskip}
\newcommand{\translation}[1]{\textsuperscript{#1}}
\newlist{algorithm}{enumerate}{10}
\setlist[algorithm]{label*=\arabic*.}
\setlist[algorithm,1]{label=\textbf{\arabic*}}
\setlist[algorithm,2]{label=\textbf{\alph*}}
\setlist[algorithm,3]{label=\textbf{\roman*}}
\setlist[algorithm,4]{label=(\arabic*)}
\setlist[algorithm,5]{label=(\alph*)}
\setlist[algorithm,6]{label=(\roman*)}
\makeatletter
\renewcommand\NAT@citesuper[3]{\ifNAT@swa
\if*#2*\else#2\NAT@spacechar\fi
\unskip\kern\p@\textsuperscript{\NAT@@open#1\if*#3*\else,\NAT@spacechar#3\fi\NAT@@close}%
   \else #1\fi\endgroup}
\makeatother
	
\begin{document}

\title{CSE3013 (컴퓨터공학 설계 및 실험 I) \space \newline CPP-1 예비 보고서}
\author{서강대학교 컴퓨터공학과 박수현 (20181634)}
\institute{서강대학교 컴퓨터공학과}
\maketitle

\section{목적}
실험에서 제시한 문제를 이해하고, 이를 해결하기 위한 알고리즘 및 자료 구조를 구상한다.

\section{문제}

\subsection{문제 이해}
\texttt{Array}, \texttt{RangeArray} 클래스를 구현하고, 동작에 맞는 멤버 함수들과 연산자들을 구현한다.

\subsection{구현 방법 구상}

\subsubsection{\mintinline{c++}{class Array}}

\begin{itemize}
	\item \mintinline{c++}{int *data} (protected): 런타임에 메모리를 할당하여 자료가 저장되는 변수이다.
	\item \mintinline{c++}{int len} (protected): 배열의 크기를 저장하는 변수이다.
	\item \mintinline{c++}{Array(int size)} (public): 크기가 \texttt{size}인 배열을 생성한다.
	\item \mintinline{c++}{~Array()} (public): 배열을 메모리에서 제거한다.
	\item \mintinline{c++}{int length()} (public): 배열의 크기를 반환한다.
	\item \mintinline{c++}{int& operator[](int i)} (public): $0 \leq \texttt{i} < \texttt{len}$일 경우, 배열의 \texttt{i}번째 위치에 값을 할당한다.
	\item \mintinline{c++}{int operator[](int i) const} (public): $0 \leq \texttt{i} < \texttt{len}$일 경우, 배열의 \texttt{i}번째 위치의 값을 반환한다.
	\item \mintinline{c++}{void print()} (public): 배열의 모든 원소를 출력한다.
\end{itemize}

또한 \mintinline{c++}{cout}은 \mintinline{c++}{std::ostream}이므로 \mintinline{c++}{print()} 대신 다음과 같은 함수를 정의하면 \mintinline{c++}{cout << array}와 같은 형식으로 배열의 모든 원소를 출력할 수 있을 것이다.
\begin{minted}[xleftmargin=\parindent,breaklines,linenos]{c}
std::ostream& operator<<(std::ostream& os, const Array &input) {
    for (int i = 0; i < input.length(); i++) {
        os << input[i] << ' ';
    }
    return os;
}
\end{minted}

\subsubsection{\mintinline{c++}{class RangeArray : public Array}}

\begin{itemize}
	\item \mintinline{c++}{int low} (protected): 배열의 시작 인덱스를 지정한다.
	\item \mintinline{c++}{int high} (protected): 배열의 끝 인덱스를 지정한다.
	\item \mintinline{c++}{RangeArray(int low, int high)} (public): 크기가 $\texttt{high} - \texttt{low} + 1$인 배열을 생성한다.
	\item \mintinline{c++}{~RangeArray()} (public): 배열을 메모리에서 제거한다.
	\item \mintinline{c++}{int baseValue()} (public): 배열의 시작 인덱스를 반환한다.
	\item \mintinline{c++}{int endValue()} (public): 배열의 끝 인덱스를 반환한다.
	\item \mintinline{c++}{int& operator[](int i)} (public): $\texttt{low} \leq \texttt{i} \leq \texttt{high}$일 경우, 배열의 \texttt{i}번째 위치에 값을 할당한다.
	\item \mintinline{c++}{int operator[](int i) const} (public): $\texttt{low} \leq \texttt{i} \leq \texttt{high}$일 경우, 배열의 \texttt{i}번째 위치의 값을 반환한다.
\end{itemize}

이 때 \mintinline{c++}{class RangeArray}는 \mintinline{c++}{class Array}를 상속받는다.

\newpage

\section{예비 학습}
\subsection{Visual Studio의 단축 키}
\begin{tabularx}{\textwidth}{l|X}
	단축 키 & 동작 \\
	\hline
	\keys{\ctrl + C} & 복사하기 \\
	\keys{\ctrl + V} & 붙여넣기 \\
	\hline
	\keys{\ctrl + F7} & 컴파일 \\
	\keys{\ctrl + F5} & 디버깅 없이 시작 \\
	\hline
	\keys{F5} & 디버깅 시작 \\
	\keys{alt + F9} & 중단점 창 \\
	\keys{F9} & 중단점 설정 / 해제 \\
	\keys{F10} & 프로시저 단위 실행 \\
	\keys{F11} & 한 단계씩 코드 실행 \\
	\keys{\ctrl + F10} & 커서까지 실행 \\
	\keys{\shift + F11} & 프로시저 나가기 \\
\end{tabularx}

\subsection{OOP}
\begin{itemize}
	\item \textbf{OOP}\translation{Object Oriented Programming}: \textbf{객체 지향 프로그래밍}이라고도 한다. OOP는 컴퓨터 프로그램을 명령어의 절차로 보기보다는 객체의 집합으로 보는 패러다임이다. 자료 추상화를 기초로 하여, 상속, 다형 개념, 동적 바인딩 등의 특징이 존재한다.
	\item \textbf{객체}\translation{object}: 변수, 자료 구조, 함수 또는 메소드가 될 수 있는 공간이다. OOP 패러다임에서는 클래스의 인스턴스를 일컫는 데에 쓰인다.
	\item \textbf{클래스}\translation{class}: 객체를 정의하기 위한 속성\translation{attribute}과 기능\translation{method}들의 집합이다. 속성과 기능들은 객체를 설계하지만, 클래스 자체가 객체인 것은 아니다. 비유하자면 클래스는 객체의 설계도인 셈이다.
	\item \textbf{인스턴스}\translation{instance}: 클래스의 정의에 따라 만들어진 객체들이다.
	\item \textbf{상속}\translation{inheritance}: 새로운 클래스가 기존의 클래스의 자료와 기능들을 사용할 수 있게 하는 기능이다. 상속을 받는 클래스를 하위 클래스, 또는 자식 클래스라고 하며 새로운 클래스가 상속하는 기존의 클래스를 상위 클래스, 또는 부모 클래스라고 한다. 
\end{itemize}

\subsection{OOP를 사용하는 이유}
\begin{itemize}
	\item 개발과 보수가 편리해지며, 추상화를 통해 직관적인 코드 분석이 가능해진다.
	\item 코드의 구조가 유연해지고 수정이 용이해지기 때문에 대규모 프로젝트에 적합하다.
	\item 라이브러리를 만들어 여러 다른 프로젝트에서 활용하기에 좋다.
\end{itemize}

\end{document}
