\documentclass[runningheads]{../../../llncs}
\usepackage[paperheight=295mm,paperwidth=210mm]{geometry}
\usepackage{graphicx}
\usepackage{import}
\usepackage{kotex}
\usepackage[dvipsnames]{xcolor}
\usepackage{fancyvrb}
\usepackage{listings}
\usepackage{indentfirst}
\usepackage{tabularx}
\usepackage{underscore}
\usepackage{multicol}
\usepackage{enumitem}
\usepackage{menukeys}
\usepackage{amsmath}
\usepackage{clrscode3e} % https://www.ctan.org/pkg/clrscode3e?lang=en
\usepackage[numbers,square,super]{natbib}
\usepackage{inconsolata} % Inconsolata
\usepackage{mathptmx} % Times New Roman
\usepackage{minted}
\graphicspath{ {./images/} }
\lstset{basicstyle=\footnotesize\ttfamily,breaklines=true}
\renewcommand{\bibname}{참고문헌}
\setlength{\parindent}{1em}
\setlength{\parskip}{1em}
\linespread{1.2}
{\renewcommand{\arraystretch}{1.5}%
\setlength{\tabcolsep}{0.5em}%
\newenvironment{Figure}
  {\par\medskip\noindent\minipage{\linewidth}}
  {\endminipage\par\medskip}
\newcommand{\translation}[1]{\textsuperscript{#1}}
\newlist{algorithm}{enumerate}{10}
\setlist[algorithm]{label*=\arabic*.}
\setlist[algorithm,1]{label=\textbf{\arabic*}}
\setlist[algorithm,2]{label=\textbf{\alph*}}
\setlist[algorithm,3]{label=\textbf{\roman*}}
\setlist[algorithm,4]{label=(\arabic*)}
\setlist[algorithm,5]{label=(\alph*)}
\setlist[algorithm,6]{label=(\roman*)}
\makeatletter
\renewcommand\NAT@citesuper[3]{\ifNAT@swa
\if*#2*\else#2\NAT@spacechar\fi
\unskip\kern\p@\textsuperscript{\NAT@@open#1\if*#3*\else,\NAT@spacechar#3\fi\NAT@@close}%
   \else #1\fi\endgroup}
\makeatother
	
\begin{document}

\title{CSE3013 (컴퓨터공학 설계 및 실험 I) \space \newline UNIX-1 결과 보고서}
\author{서강대학교 컴퓨터공학과 박수현 (20181634)}
\institute{서강대학교 컴퓨터공학과}
\maketitle

\section{소스 코드}
\textbf{phone}
\inputminted[xleftmargin=\parindent,breaklines,linenos]{shell}{inc-sources/phone}

\textbf{display.awk}
\inputminted[xleftmargin=\parindent,breaklines,linenos]{awk}{inc-sources/display.awk}
  
\section{and, or 동시 지원}
1번째 argument에 and 혹은 or를 명시하는 옵션을 달아 다음과 같은 명령어 구조를 사용할 수 있을 것이다.

\begin{minted}[xleftmargin=\parindent,breaklines,linenos]{shell}
./phone [-and|-or] [queries...]
\end{minted}


그러면 스크립트에서 1번째 인자에 따라 and-mode와 or-mode 중 하나를 선택해 실행할 수 있을 것이다. \texttt{if}-\texttt{elif}-\texttt{else}-\texttt{fi} 컨트롤 플로우를 사용하면 된다. 따라서 다음과 같은 코드 구조를 생각할 수 있다.

\begin{minted}[xleftmargin=\parindent,breaklines,linenos]{shell}
if [ $1='-and']; then
    # and-mode
elif [ $1='-or' ]; then
    # or-mode
else
    # error handling
fi
\end{minted}

\end{document}
