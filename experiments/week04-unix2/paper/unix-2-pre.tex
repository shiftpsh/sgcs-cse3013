\documentclass[runningheads]{../../../llncs}
\usepackage[paperheight=295mm,paperwidth=210mm]{geometry}
\usepackage{graphicx}
\usepackage{import}
\usepackage{kotex}
\usepackage[dvipsnames]{xcolor}
\usepackage{fancyvrb}
\usepackage{listings}
\usepackage{indentfirst}
\usepackage{tabularx}
\usepackage{underscore}
\usepackage{multicol}
\usepackage{enumitem}
\usepackage{menukeys}
\usepackage{amsmath}
\usepackage{clrscode3e} % https://www.ctan.org/pkg/clrscode3e?lang=en
\usepackage[numbers,square,super]{natbib}
\usepackage{inconsolata} % Inconsolata
\usepackage{mathptmx} % Times New Roman
\usepackage{minted}
\graphicspath{ {./images/} }
\lstset{basicstyle=\footnotesize\ttfamily,breaklines=true}
\renewcommand{\bibname}{참고문헌}
\setlength{\parindent}{1em}
\setlength{\parskip}{1em}
\linespread{1.2}
{\renewcommand{\arraystretch}{1.5}%
\setlength{\tabcolsep}{0.5em}%
\newenvironment{Figure}
  {\par\medskip\noindent\minipage{\linewidth}}
  {\endminipage\par\medskip}
\newcommand{\translation}[1]{\textsuperscript{#1}}
\newlist{algorithm}{enumerate}{10}
\setlist[algorithm]{label*=\arabic*.}
\setlist[algorithm,1]{label=\textbf{\arabic*}}
\setlist[algorithm,2]{label=\textbf{\alph*}}
\setlist[algorithm,3]{label=\textbf{\roman*}}
\setlist[algorithm,4]{label=(\arabic*)}
\setlist[algorithm,5]{label=(\alph*)}
\setlist[algorithm,6]{label=(\roman*)}
\makeatletter
\renewcommand\NAT@citesuper[3]{\ifNAT@swa
\if*#2*\else#2\NAT@spacechar\fi
\unskip\kern\p@\textsuperscript{\NAT@@open#1\if*#3*\else,\NAT@spacechar#3\fi\NAT@@close}%
   \else #1\fi\endgroup}
\makeatother
	
\begin{document}

\title{CSE3013 (컴퓨터공학 설계 및 실험 I) \space \newline UNIX-2 예비 보고서}
\author{서강대학교 컴퓨터공학과 박수현 (20181634)}
\institute{서강대학교 컴퓨터공학과}
\maketitle

\section{목적}
UNIX 상에서 제공하는 C/C++ 관련 도구를 미리 사용해 봄으로써 실습이 원활히 진행될 수 있도록 한다.

\section{예비 학습}
C/C++ 소스 파일의 빌드 과정은 \textbf{프리프로세싱}\translation{preprocessing}, \textbf{컴파일링}\translation{compiling}, \textbf{어셈블링}\translation{assembling}, \textbf{링킹}\translation{linking}으로 구성된다.

\subsection{프리프로세싱}
프리프로세싱은 선행 처리기라고도 불리는 \textbf{프리프로세서}\translation{preprocessor}에 의해 이루어진다. 이름에서 보이듯이, 코드를 컴파일러가 처리하기 전에 특정 변수를 미리 정의된 문자열로 치환하는 등의 작업을 수행한다.

C/C++에서는 매크로\translation{macro}와 매크로 확장\translation{macro expansion}을 정의할 수 있다. 사용예는 다음과 같다.

\begin{minted}[xleftmargin=\parindent,breaklines,linenos]{c}
#include <stdio.h>
#define PI 3.141592653
#define area(x) (PI * (x) * (x))

int main() {
    printf("%f\n", area(10.0f));
    return 0;
}
\end{minted}

이 코드는 프리프로세서에 의해 컴파일러가 처리하기 전에 다음과 같이 바뀐다.

\begin{minted}[xleftmargin=\parindent,breaklines,linenos]{c}
#include <stdio.h>

int main() {
    printf("%f\n", (3.141592653 * (10.0f) * (10.0f)));
    return 0;
}
\end{minted}

\subsection{컴파일링}
컴파일은 \textbf{컴파일러}\translation{compiler}에 의해 이루어진다. C/C++ 소스 파일은 컴파일러를 통해 어셈블리 소스 파일이 된다. '컴파일'은 이 과정이지만, 일반적으로 어셈블링과 묶여서, 혹은 소스가 실행 파일이 되는 전 과정을 묶어서 '컴파일'이라고 부르기도 한다. 대표적인 컴파일러들로는 gcc \translation{GNU Compiler Collection}, LLVM, 그리고 MS Visual C++ compiler가 있는데, 이들은 프리프로세서의 역할도 한다.

컴파일 과정은 전단부\translation{front-end}, 중단부\translation{middle-end}, 후단부\translation{back-end}로 구성된다. 이 전 과정이 실행되는 시간을 \textbf{컴파일 타임}\translation{compile time}이라고 한다.

전단부에서는 \textbf{렉서}\translation{lexer}(어휘 해석기)가 소스코드를 토큰\translation{token}으로 나누면서 문법적 오류를 검출하고, 이 토큰들로 \textbf{파서}\translation{parser}(구문 해석기)가 파스 트리\translation{parse tree}를 구성하여 기능적 오류를 검출한다. 마지막으로 언어에 비종속적인 GIMPLE 트리를 만들어 중단부에 제공한다.

중단부에서는 GIMPLE 트리를 SSA\translation{Static Single Assignment} 형식의 자료로 변환한 후, 환경에 종속적이지 않은 최적화를 진행한다. 여기서 환경이랑 OS 혹은 CPU 아키텍쳐를 말한다. 최적화가 완료되면 후단부에 RTL\translation{Register Transfer Language} 형식으로 제공한다.

후단부에서는 RTL 최적화 이후 환경에 종속적인 최적화를 수행한다. 최적화가 완료되면 코드 생성기가 어셈블리 코드를 생성하고, 최종적으로 어셈블러에 코드를 넘겨주게 된다.

gcc의 경우 \texttt{-O0, -O1, -O2, -O3, -Os, -Og, -Ofast} 등의 플래그로 최적화 단계를 지정할 수 있다.

\subsection{어셈블링}
컴파일된 코드는 \textbf{어셈블러}\translation{assembler}가 한 번 더 처리한다. 이 때 어셈블러가 생성하는 파일을 목적 코드\translation{object code}라고 한다. 이 목적 파일은 명령어와 자료가 저장된 ELF 바이너리 포맷이다. 

어셈블러는 주로 어셈블리 니모닉\translation{mnemonics}을 op-code로 해석하는 작업을 하며, 컴파일러보다 구조가 훨씬 간단하다. 어셈블러가 실행되는 시간을 \textbf{어셈블리 타임}\translation{assembly time}이라고 한다.

\subsection{링킹}
프리프로세싱, 컴파일링, 어셈블링을 거친 코드는 마지막으로 \textbf{링커}\translation{linker}에 의해 라이브러리와 링킹된다. 링커는 목적 파일들과 프로그래머가 프로그램에서 명시한 표준 C 라이브러리(\texttt{stdio.h} 따위의), 사용자 라이브러리 등을 링크\translation{link}한다. 만들어진 목적 파일들은 링킹 과정을 거치면서 최종적으로 OS가 런타임에서 실행할 수 있는 실행 파일\translation{executable file}이 된다.

\end{document}
