\documentclass[runningheads]{../../../llncs}
\usepackage[paperheight=295mm,paperwidth=210mm]{geometry}
\usepackage{graphicx}
\usepackage{import}
\usepackage{kotex}
\usepackage[dvipsnames]{xcolor}
\usepackage{fancyvrb}
\usepackage{listings}
\usepackage{indentfirst}
\usepackage{tabularx}
\usepackage{underscore}
\usepackage{multicol}
\usepackage{enumitem}
\usepackage{menukeys}
\usepackage{amsmath}
\usepackage{clrscode3e} % https://www.ctan.org/pkg/clrscode3e?lang=en
\usepackage[numbers,square,super]{natbib}
\usepackage{inconsolata} % Inconsolata
\usepackage{mathptmx} % Times New Roman
\usepackage{minted}
\graphicspath{ {./images/} }
\lstset{basicstyle=\footnotesize\ttfamily,breaklines=true}
\renewcommand{\bibname}{참고문헌}
\setlength{\parindent}{1em}
\setlength{\parskip}{1em}
\linespread{1.2}
{\renewcommand{\arraystretch}{1.5}%
\setlength{\tabcolsep}{0.5em}%
\newenvironment{Figure}
  {\par\medskip\noindent\minipage{\linewidth}}
  {\endminipage\par\medskip}
\newcommand{\translation}[1]{\textsuperscript{#1}}
\newlist{algorithm}{enumerate}{10}
\setlist[algorithm]{label*=\arabic*.}
\setlist[algorithm,1]{label=\textbf{\arabic*}}
\setlist[algorithm,2]{label=\textbf{\alph*}}
\setlist[algorithm,3]{label=\textbf{\roman*}}
\setlist[algorithm,4]{label=(\arabic*)}
\setlist[algorithm,5]{label=(\alph*)}
\setlist[algorithm,6]{label=(\roman*)}
\makeatletter
\renewcommand\NAT@citesuper[3]{\ifNAT@swa
\if*#2*\else#2\NAT@spacechar\fi
\unskip\kern\p@\textsuperscript{\NAT@@open#1\if*#3*\else,\NAT@spacechar#3\fi\NAT@@close}%
   \else #1\fi\endgroup}
\makeatother
	
\begin{document}

\title{CSE3013 (컴퓨터공학 설계 및 실험 I) \space \newline CPP-2 예비 보고서}
\author{서강대학교 컴퓨터공학과 박수현 (20181634)}
\institute{서강대학교 컴퓨터공학과}
\maketitle

\section{목적}
실험과 과제에서 제시한 문제를 이해하고, 이를 해결하기 위해 사용한 알고리즘 및 자료 구조를 기술한다.

\section{문제}

\subsection{실험}

\mintinline{c++}{template <class T> class Node}
\begin{itemize}
	\item \mintinline{c++}{T data} (public): 노드의 값을 저장하는 변수이다.
	\item \mintinline{c++}{Node *link} (public): 다음 노드의 참조를 저장하는 변수이다.
\end{itemize}

\mintinline{c++}{template <class T> class LinkedList}
\begin{itemize}
	\item \mintinline{c++}{Node<T> *first} (protected): 링크드 리스트의 첫 노드의 참조를 저장하는 변수이다.
	\item \mintinline{c++}{int current_size} (protected): 링크드 리스트의 크기를 저장하는 변수이다.
	\item \mintinline{c++}{LinkedList()} (public): 링크드 리스트의 생성자이다.
	\item \mintinline{c++}{int GetSize()} (public): 링크드 리스트의 크기(=\texttt{current_size})를 반환하는 함수이다.
	\item \mintinline{c++}{void Insert(T element)} (public): 링크드 리스트의 \textbf{맨 앞}에 원소를 삽입하는 함수이다.
	노드를 하나 만들고 새로 만든 노드 다음 노드의 참조를 \mintinline{c++}{*first}로 설정한 뒤, 링크드 리스트의 첫 번째 노드를 새로 만든 노드로 설정한다.
	\item \mintinline{c++}{virtual bool Delete(T &element)} (public): 링크드 리스트의 \textbf{마지막} 원소를 제거한다.
	링크드 리스트의 노드를 끝까지 탐색한 후 마지막 노드를 제거하고, \texttt{current_size}를 1 줄인다. 모든 노드를 탐색하는 데 걸리는 시간은
	노드가 $n$개일 때 $\mathcal{O}\left(n\right)$이다.
	\item \mintinline{c++}{void Print()} (public): 링크드 리스트의 모든 원소를 차례로 출력한다. 모든 노드를 탐색하면서 출력하므로
	노드가 $n$개일 때 시간 복잡도는 $\mathcal{O}\left(n\right)$이다.
\end{itemize}

\subsection{과제}

\mintinline{c++}{template <class T> class Array}
\begin{itemize}
	\item \mintinline{c++}{T *data} (protected): 배열의 값들을 저장하는 변수이다.
	\item \mintinline{c++}{int len} (protected): 배열의 길이를 저장하는 변수이다.
	\item \mintinline{c++}{Array(){}} (public): 배열 생성자이다.
	\item \mintinline{c++}{Array(int size)} (public): 크기가 \texttt{size}인 배열을 생성한다.
	\item \mintinline{c++}{~Array()} (public): 배열을 메모리에서 해제한다.
	\item \mintinline{c++}{int length() const} (public): 배열의 크기(=\texttt{len})를 반환하는 함수이다.
	\item \mintinline{c++}{virtual T &operator[](int i)} (public): 배열의 \texttt{i}번째 원소의 참조를 반환한다.
	\texttt{i}가 인덱스 범위에서 벗어날 경우에는 오류 메시지를 출력한다.
	메모리 주소를 액세스하므로 시간 복잡도는 $\mathcal{O}\left(1\right)$이다.
	\item \mintinline{c++}{virtual T operator[](int i) const} (public): 배열의 \texttt{i}번째 원소의 값을 반환한다.
	\texttt{i}가 인덱스 범위에서 벗어날 경우에는 오류 메시지를 출력한다.
	메모리 주소를 액세스하므로 시간 복잡도는 $\mathcal{O}\left(1\right)$이다.
	\item \mintinline{c++}{void print()} (public): 배열 내의 모든 원소를 출력한다.
	원소가 $n$개일 때 시간 복잡도는 $\mathcal{O}\left(n\right)$이다.
\end{itemize}

\mintinline{c++}{template <class T> class GrowableArray : public Array<T>}
\begin{itemize}
	\item \mintinline{c++}{GrowableArray(int size)} (public): 크기 가변 배열의 생성자이다. 크기가 \texttt{size}인 배열을 생성한다.
	\item \mintinline{c++}{virtual T &operator[](int i)} (public): 배열의 \texttt{i}번째 원소의 참조를 반환한다.
	\texttt{i}가 0보다 작을 경우에는 오류 메시지를 출력하고, 
	\texttt{i}가 인덱스 범위에서 벗어날 경우에는 배열의 크기를 $2\texttt{i}$로 설정한다.
	\texttt{i}가 인덱스 범위 안에 있을 경우 시간 복잡도는 $\mathcal{O}\left(1\right)$이고, 밖에 있을 경우 시간 복잡도는 $\mathcal{O}\left(i\right)$이다.
\end{itemize}

\end{document}
