\subsubsection{테트리스 훈련 프로그램}
테트리스 훈련 프로그램은 JVM 언어인 Kotlin(.kt)과 Java(.java)로 구성되어 있다. 빌드를 위해 Gradle을 사용했으며, build.gradle은 Gradle 빌드 스크립트이다.
전체 프로젝트는 다음과 같은 파일로 구성된다.

\begin{tabularx}{\linewidth}{l|X}
    파일 이름 & 기능 \\
    \hline
    com.shiftpsh.tetris\_ai.main.kt & 훈련 진행 \\
    com.shiftpsh.tetris\_ai.constraints.kt & 계수 벡터에 대한 데이터 클래스 정의 \\
    com.shiftpsh.tetris\_ai.result.kt & 실행 결과에 대한 데이터 클래스 정의 \\
    \hline
    com.shiftpsh.tetris\_ai.TetrisJNI.java & 네이티브 라이브러리로 포팅한 테트리스 프로그램과 Java 훈련 프로그램 사이의 JNI 인터페이스 \\
    \hline
    libtetris.so & Linux x86\_64에서 빌드한, JNI로 포팅된 테트리스 프로그램의 네이티브 라이브러리 \\
    libtetris.dylib & macOS x86\_64에서 빌드한, JNI로 포팅된 테트리스 프로그램의 네이티브 라이브러리 \\
    \hline
    build.gradle & 프로젝트의 빌드 정보 \\
\end{tabularx}

com.shiftpsh.tetris\_ai.main.kt에는 다음과 같은 메서드들이 있다.
\begin{itemize}
    \item \mintinline[breaklines]{kotlin}{fun execute(constraints: Constraints): Result}
    테트리스 게임을 추천 모드로 가중치 \texttt{constraints}를 적용해 한 번 실행한다.
    \item \mintinline[breaklines]{kotlin}{fun batch_execute(iterations: Int, constraints: Constraints): Result}
    테트리스 게임을 추천 모드로 가중치 \texttt{constraints}를 적용해 \texttt{iterations}번 실행하고, 결과값의 12.5\% 절사평균을 구한다.
    \item \mintinline[breaklines]{kotlin}{fun main()): Result}
    훈련을 실행한다.
\end{itemize}

com.shiftpsh.tetris\_ai.constraints.kt와 com.shiftpsh.tetris\_ai.result.kt에는 각각 하나의 데이터 클래스가 정의되어 있다.

\mintinline[breaklines]{kotlin}{fun execute(constraints: Constraints): Result} 함수의 코드는 다음과 같다.

\inputminted[xleftmargin=\parindent,linenos,firstline=9,lastline=16,breaklines]{kotlin}{inc-sources-java/main.kt}

이 함수는 \texttt{TetrisJNI} 클래스의 \texttt{simulate} 함수를 호출한다. 아래는 \texttt{TetrisJNI} 클래스의 전체 코드이다.

\inputminted[xleftmargin=\parindent,linenos,breaklines]{java}{inc-sources-java/TetrisJNI.java}

\mintinline[breaklines]{java}{System.loadLibrary("tetris")}는 환경에 따라 필요한 네이티브 라이브러리를 로드한다. macOS의 경우 libtetris.dylib, 
Linux의 경우 libtetris.so와 같은 식이다. 여기에서 \mintinline[breaklines]{java}{private native String run(String[] constraints)}의 구현은
찾아볼 수 없는데, \mintinline[breaklines]{java}{native} 키워드는 이 함수가 네이티브 라이브러리에 구현되어 있음을 의미한다.

\mintinline[breaklines]{shell}{javac -h . TetrisJNI.java}를 통해 C 프로그램을 JNI로 포팅하기 위한 헤더 파일 템플릿을 생성할 수 있다.\cite{JNI}
생성된 헤더 파일에는 다음과 같은 코드가 있는데, 이를 구현해 주면 된다.

\begin{minted}{c}
/*
 * Class:     com_shiftpsh_tetris_ai_TetrisJNI
 * Method:    run
 * Signature: ([Ljava/lang/String;)Ljava/lang/String;
 */
JNIEXPORT jstring JNICALL Java_com_shiftpsh_tetris_1ai_TetrisJNI_run
  (JNIEnv *, jobject, jobjectArray);
\end{minted}

이 형식들은 JDK의 jni.h에 정의된 형식들이다. \texttt{jobjectArray}가 \mintinline[breaklines]{java}{String[] constraints}를 의미한다.

ncurses 기능을 전부 제거한 C 파일에는 다음과 같이 구현한다.

\begin{minted}[breaklines]{c}
JNIEXPORT jstring JNICALL Java_com_shiftpsh_tetris_1ai_TetrisJNI_run
        (JNIEnv *env, jobject object, jobjectArray stringArray) {
    for (int i = 0; i < 18; i++) {
        feature_weights[i] = strtold((*env)->GetStringUTFChars(env, (jstring) ((*env)->GetObjectArrayElement(env, stringArray, 0)), 0), NULL);
    }

    srand((unsigned int)time(NULL));
    recommendedPlay();

    char buffer[1000];
    sprintf(buffer, "%d %d %lf\n\n", score, droppedBlocks, (double) score / droppedBlocks);

    return (*env)->NewStringUTF(env, buffer);
}
\end{minted}

이 함수는 Java 문자열의 배열을 받아 C 문자열로 해석한 뒤에 이를 \texttt{long double}로 해석하고, 이를 가중치 벡터로 적용하여
프로그램을 실행한 뒤 실행이 끝나면 점수와 게임 오버 직전까지 떨어뜨린 블록 수와 블록 당 점수를 문자열로 반환한다. 이렇게 작성된 C 파일들은

\mintinline[breaklines]{shell}{gcc -fPIC -I"$JAVA_HOME/include" -I"$JAVA_HOME/include/linux" -shared -o libtetris.so TetrisJNI.c}

명령어를 통해 네이티브 라이브러리로 컴파일할 수 있다.\cite{JNI}

정리하면, 이 프로그램은 Kotlin에서 \mintinline[breaklines]{kotlin}{execute}를 호출하면 Java에서
\mintinline[breaklines]{java}{simulate}가 호출되고, 이는 다시 JNI를 통해
\mintinline[breaklines]{c}{Java_com_shiftpsh_tetris_1ai_TetrisJNI_run}을
호출하는 구조이다.

실제로 최적의 값을 찾는 알고리즘은 다음과 같다.

\inputminted[xleftmargin=\parindent,linenos,firstline=27,lastline=63,breaklines]{kotlin}{inc-sources-java/main.kt}

이는 현재까지 최적의 벡터를 평균으로 하고 분산을 50으로 하는 가우시안 분포에서 주변의 벡터를 하나 고른다. 이 벡터를 바탕으로 프로그램을 16번 실행하고 얻은 결과의
12.5\% 절사평균을 구한다. 마지막으로 설계한 대로 확률 $p$를 산출해, 이 확률에 따라 값을 계속 업데이트하면서 진행한다.

$\mathbf{0}$으로 초기값을 설정하면 훈련이 오래 걸릴 것을 예상해, 시행착오를 통해 얻은 다음 벡터를 초기값으로 설정하였다.

\begin{align*}
    \mathbf{x}_0 = 100 \left<\right.& 25,\,150,\,-45,\,25,\,-100,\,-180, \\
    & 160,\,190,\,-24,\,25,\,30,\,60,\\
    & 80,\,-240,\,20,\,96,\,75 \left.\right>
\end{align*}

4줄을 동시에 없애는 휴리스틱에 대한 가중치는 항상 $-10^9$로 고정시켰다.

이 프로그램을 빌드하는 데 사용한 빌드 스크립트의 내용은 다음과 같다.

\inputminted[xleftmargin=\parindent,linenos,breaklines]{text}{inc-sources-java/build.gradle}

마지막의 \texttt{jar} 동작은 프로그램을 JAR executable로 빌드한다. 이렇게 빌드된 바이너리는
어느 운영 체제에서나 \texttt{java -jar tetris_ai.jar}와 같은 식으로 실행할 수 있다.

이렇게 만들어진 바이너리를 AWS EC2 인스턴스에서 실행시켜 봤으나, 여러 문제가 발생하였다.

첫째는 \texttt{ThreadPool}을 이용해 여러 스레드를 만들어 CPU의 자원을 최대한
활용하려 했으나, 어째서인지 단일 스레드로 높은 점수를 내는 벡터가 다중 스레드에서는
20블록도 넘기지 못하고 게임 오버를 당하는 것이었다. 정확한 원인을 찾지 못해 결국 단일 스레드만을
활용하도록 프로그램을 수정해야 했다.

둘째는 프로그램의 실행 시간이 너무 오래 걸린다는 것이었다. 초기 벡터의 경우만 하더라도
약 5,000블록을 진행하고도 게임 오버가 되지 않는 경우가 있었다. 이 문제는 훈련 목표에서
게임 오버 직전까지 떨어뜨린 블록 수를 빼고 순수히 블록 당 얻은 점수로만 설정함으로써
해결하였다.

유감스럽게도 프로젝트를 진행할 수 있는 시간이 한정되어 있어서 훈련 과정을 제대로
진행하지 못했다. 마지막으로 얻은 최적해는 다음과 같았다.

\begin{align*}
    \mathbf{x} = \left<\right.& 2495,\,14835,\,-4424,\,2510,\,-10355,\,-18376, \\
    & 16388,\,18910,\,-2315,\,2528,\,3160,\,6203,\\
    & 7942,\,-23070,\,2111,\,9615,\,7624 \left.\right>
\end{align*}

\newpage