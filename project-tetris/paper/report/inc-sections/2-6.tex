Figure \ref{fig:gameplay}은 보통 플레이 모드의 스크린샷이다. 블록 조작, 스코어 가산,
그림자, 추천 기능, 다음 블록을 생성하고 보여주는 기능 등 핵심적인 기능들이 모두 잘 동작하였다.

Figure \ref{fig:gameover}은 게임 오버 스크린샷이다. 게임 오버 조건을 만족하였을 때
정상적으로 게임이 종료되었다.

Figure \ref{fig:endgame-results}은 게임 오버 이후 랭킹을 입력하면 보이는 화면의
스크린샷이다. 랭킹 정보를 문제 없이 읽고 쓸 수 있었다.

Figure \ref{fig:rankings}은 랭킹 기능에서 이름을 기준으로 쿼리한 결과의 스크린샷이다.
랭킹 범위로 검색, 랭킹 삭제, 이름으로 쿼리 모두 정상적으로 작동하였다.

마지막으로 Figure \ref{fig:recommend-play}은 추천 플레이 기능의 스크린샷이다.
추천 플레이 기능은 미래 블록 3개를 고려할 때 초당 약 60개의 블록을 떨어뜨리도록
수정되었다.

추천 플레이 기능은 잘 동작하였으나, 점수 표시 칸 오른쪽에 알 수 없는 잔상이 남았다.
오른쪽의 통계를 그리느라 남은 잔상으로 추정되나 정확한 원인을 규명할 수 없어서 해결하지 못했다.

Figure \ref{fig:best-play}는 가장 오래 살아남은 추천 플레이의 결과 스크린샷이다.
3블록 앞을 볼 때 가장 오래 살아남았을 경우 149,997블록을 떨어뜨리고 게임오버되었으며,
점수는 17,513,490점이었다. 초당 54개의 블록을 떨어뜨렸고 총 실행 시간은 46.1분이었다.

추천 알고리즘의 복잡도와 효율성을 검증하기 위해 다음과 같은 평가를 수행하였다.
\begin{itemize}
    \item 다른 조건은 그대로 두고, 미래에 보는 블록 수에 따라 시뮬레이션을 12번 진행해
    추천 알고리즘의 누적 계산 시간과 누적 사용 공간, 점수의
    평균과 표준편차를 각각 구한다.
    \item 위 평가를 트리 가지치기를 하지 않고 진행한다.
    \item 다른 조건은 그대로 두고, 트리 가지치기 이후에 남기는 노드의 상한에 따라
    시뮬레이션을 12번 진행해 추천 알고리즘의 누적 계산 시간과 누적 사용 공간, 점수의
    평균과 표준편차를 각각 구한다.
\end{itemize}
단, 프로그램이 쉽게 게임 오버되지 않는 점을 고려해 1,000번째 블록을 떨어뜨리는 시점에서
강제로 게임이 종료되도록 한다. 또한 1,000번째 블록 이전에서 게임 오버를 당할 경우
1,000블록 당 시간, 공간, 점수를 계산한다.

1번째 평가의 결과는 다음과 같았다. 남기는 노드의 상한은 모든 경우에서 32였다.
\begin{center}
    \begin{tabular}{l|r|r|r|r|r}
        미래에 보는 블록 수 & 1* & 2 & 3 & 4 & 5 \\
        \hline
        점수 평균 & 99,470 & 118,352 & 130,272 & 136,363 & 137,178 \\
        점수 표준편차 & 11,317 & 8,675 & 7,253 & 5,806 & 7,357 \\
        \hline
        누적 계산 시간 평균 (초) & $<0.01$ & 3.80 & 7.26 & 16.71 & 26.04 \\
        누적 계산 시간 표준편차 (초) & $<0.01$ & 0.14 & 0.25 & 0.43 & 0.80 \\
        1초당 점수 & $\infty$ & 31,145 & 17,944 & 8,161 & 5,268 \\
        \hline
        누적 사용 메모리 평균 (MB) & 7.29 & 168.31 & 397.50 & 630.41 & 856.28 \\
        누적 사용 메모리 표준편차 (MB) & 0.22 & 3.82 & 4.97 & 12.44 & 12.78 \\
        1MB당 점수 & 13,645 & 703 & 327 & 216 & 160 \\
    \end{tabular}
\end{center}
* 미래에 1블록만 보는 경우는 평균적으로 456번째 블록이 내려온 직후 게임 오버 당했다.
나머지 경우에서는 1,000번째 블록이 내려올 때까지 게임 오버를 당하지 않았다.

1초당 점수와 1MB당 점수는 미래에 보는 블록 수가 증가할수록 오히려 감소했지만,
미래에 1블록만 보는 경우에서 1,000번째 블록이 내려올 때까지 게임 오버를 당하지 않은
경우가 없었던 것으로 미루어 보아 미래에 많은 블록을 볼수록 안정성이 높아진다고
추측할 수 있다. 따라서 안정성과 효율의 균형이 잘 맞는 블록 수를 선택해야 할 것이다.

점수 평균의 그래프는 다음과 같았다.

\begin{center}
    \begin{tikzpicture}[domain=0:6]
        \begin{axis}[xlabel={미래에 보는 블록 수}, ylabel={점수 평균}]
        \addplot[scatter, scatter src=\thisrow{class},
              error bars/.cd, y dir=both, x dir=both, y explicit, x explicit, error bar style={color=mapped color}]
              table[x=x,y=y,x error=xerr,y error=yerr] {
            x xerr y yerr class
            1 0 99470 11317 0
            2 0 118352 8675 0
            3 0 130272 7253 0
            4 0 136363 5806 0
            5 0 137178 7347 0
        };
        \end{axis}
    \end{tikzpicture}
\end{center}

미래에 보는 블록 수가 증가할수록 어떤 값에 수렴하는 형태를 보인다.

누적 계산 시간 평균의 그래프는 다음과 같았다.

\begin{center}
    \begin{tikzpicture}[domain=0:6]
        \begin{axis}[xlabel={미래에 보는 블록 수}, ylabel={누적 계산 시간(초)}]
        \addplot[scatter, scatter src=\thisrow{class},
              error bars/.cd, y dir=both, x dir=both, y explicit, x explicit, error bar style={color=mapped color}]
              table[x=x,y=y,x error=xerr,y error=yerr] {
            x xerr y yerr class
            1 0 0 0 0
            2 0 3.80 0.14 0
            3 0 7.26 0.25 0
            4 0 16.71 0.43 0
            5 0 26.04 0.80 0
        };
        \end{axis}
    \end{tikzpicture}
\end{center}

1--3블록, 3--5블록은 각각 선형의 경향성을 보이고 있다. 3블록을 기점으로
누적 계산 시간이 꺾였는데, 요인은 잦은 캐시 미스 혹은 소팅 시간의 증가라고 추측된다.

누적 사용 메모리 평균의 그래프는 다음과 같았다.

\begin{center}
    \begin{tikzpicture}[domain=0:6]
        \begin{axis}[xlabel={미래에 보는 블록 수}, ylabel={누적 사용 메모리(MB)}]
        \addplot[scatter, scatter src=\thisrow{class},
              error bars/.cd, y dir=both, x dir=both, y explicit, x explicit, error bar style={color=mapped color}]
              table[x=x,y=y,x error=xerr,y error=yerr] {
            x xerr y yerr class
            1 0 7.29 0.22 0
            2 0 168.31 3.82 0
            3 0 397.50 4.97 0
            4 0 630.41 12.44 0
            5 0 856.28 12.78 0
        };
        \end{axis}
    \end{tikzpicture}
\end{center}

선형의 경향성을 보이고 있다. 이는 가지치기를 했을 때 공간 복잡도가 미래에
보는 블록의 수에 비례함을 뒷받침한다.

2번째 평가의 결과는 다음과 같았다.
\begin{center}
    \begin{tabular}{l|r|r|r|r|r}
        미래에 보는 블록 수 & 1* & 2 & 3 & 4** & 5** \\
        \hline
        점수 평균 & 97,456 & 114,163 & 126,863 & -- & -- \\
        점수 표준편차 & 15,523 & 5,958 & 5,383 & -- & -- \\
        \hline
        누적 계산 시간 평균 (초) & $<0.01$ & 5.33 & 96.78 & -- & -- \\
        누적 계산 시간 표준편차 (초) & $<0.01$ & 0.22 & 4.36 & -- & -- \\
        1초당 점수 & $\infty$ & 21,479 & 1,311 & -- & -- \\
        \hline
        누적 사용 메모리 평균 (MB) & 7.40 & 176.72 & 4349.17 & -- & -- \\
        누적 사용 메모리 표준편차 (MB) & 0.17 & 3.52 & 197.27 & -- & -- \\
        1MB당 점수 & 13,170 & 646 & 29 & -- & -- \\
    \end{tabular}
\end{center}
* 미래에 1블록만 보는 경우는 평균적으로 477번째 블록이 내려온 직후 게임 오버 당했다.
나머지 경우에서는 1,000번째 블록이 내려올 때까지 게임 오버를 당하지 않았다.

** 미래에 4블록, 5블록을 보는 경우는 1,000번째 블록이 떨어질 때까지의 시간이
너무 오래 걸려 테스트하지 못했다.

가지치기를 하지 않을 경우 사용하는 시간과 공간이 기하급수적으로 늘어남을 알 수 있다.

점수 평균의 그래프는 다음과 같았다. 붉은색 점은 가지치기를 했을 때의 결과이다.

\begin{center}
    \begin{tikzpicture}[domain=0:6]
        \begin{axis}[xlabel={미래에 보는 블록 수}, ylabel={점수 평균}]
        \addplot[scatter, scatter src=\thisrow{class},
              error bars/.cd, y dir=both, x dir=both, y explicit, x explicit, error bar style={color=mapped color}]
              table[x=x,y=y,x error=xerr,y error=yerr] {
            x xerr y yerr class
            1 0 97456 15523 0
            2 0 114163 5958 0
            3 0 126863 5383 0
        };
        \addplot[scatter, scatter src=\thisrow{class},
              error bars/.cd, y dir=both, x dir=both, y explicit, x explicit, error bar style={color=mapped color}]
              table[x=x,y=y,x error=xerr,y error=yerr] {
            x xerr y yerr class
            1 0 99470 11317 1
            2 0 118352 8675 1
            3 0 130272 7253 1
            4 0 136363 5806 1
            5 0 137178 7347 1
        };
        \end{axis}
    \end{tikzpicture}
\end{center}

가지치기를 했을 때의 결과가 약간 더 좋았으나 점수 평균에서는 의미 있는 차이가 보이지 않았다.

누적 계산 시간 평균의 그래프는 다음과 같았다. 붉은색 점은 가지치기를 했을 때의 결과이다.

\begin{center}
    \begin{tikzpicture}[domain=0:6]
        \begin{axis}[xlabel={미래에 보는 블록 수}, ylabel={누적 계산 시간(초)}]
        \addplot[scatter, scatter src=\thisrow{class},
              error bars/.cd, y dir=both, x dir=both, y explicit, x explicit, error bar style={color=mapped color}]
              table[x=x,y=y,x error=xerr,y error=yerr] {
            x xerr y yerr class
            1 0 0 0 0
            2 0 5.33 0.22 0
            3 0 96.78 4.36 0
        };
        \addplot[scatter, scatter src=\thisrow{class},
              error bars/.cd, y dir=both, x dir=both, y explicit, x explicit, error bar style={color=mapped color}]
              table[x=x,y=y,x error=xerr,y error=yerr] {
            x xerr y yerr class
            1 0 0 0 1
            2 0 3.80 0.14 1
            3 0 7.26 0.25 1
            4 0 16.71 0.43 1
            5 0 26.04 0.80 1
        };
        \end{axis}
    \end{tikzpicture}
\end{center}

가지치기를 하지 않았을 때 미래에 3블록을 보는 경우 기하급수적으로 시간이 늘어났음을
확인할 수 있다.

누적 사용 메모리 평균의 그래프는 다음과 같았다. 붉은색 점은 가지치기를 했을 때의 결과이다.

\begin{center}
    \begin{tikzpicture}[domain=0:6]
        \begin{axis}[xlabel={미래에 보는 블록 수}, ylabel={누적 사용 메모리(MB)}]
        \addplot[scatter, scatter src=\thisrow{class},
              error bars/.cd, y dir=both, x dir=both, y explicit, x explicit, error bar style={color=mapped color}]
              table[x=x,y=y,x error=xerr,y error=yerr] {
            x xerr y yerr class
            1 0 7.40 0.17 0
            2 0 176.72 3.52 0
            3 0 4349.17 197.27 0
        };
        \addplot[scatter, scatter src=\thisrow{class},
              error bars/.cd, y dir=both, x dir=both, y explicit, x explicit, error bar style={color=mapped color}]
              table[x=x,y=y,x error=xerr,y error=yerr] {
            x xerr y yerr class
            1 0 7.29 0.22 1
            2 0 168.31 3.82 1
            3 0 397.50 4.97 1
            4 0 630.41 12.44 1
            5 0 856.28 12.78 1
        };
        \end{axis}
    \end{tikzpicture}
\end{center}

가지치기를 하지 않았을 때 미래에 3블록을 보는 경우 기하급수적으로 사용 메모리가 늘어났음을
확인할 수 있다. 그래프들을 종합해 보면 가지치기는 시간, 공간적 측면 모두에서 결정 트리를
효율적으로 사용할 수 있게 해 주는 수단이라는 것을 확인할 수 있다.

마지막으로 2번째 평가의 결과는 다음과 같았다. 미래에 보는 블록의 수는 3개로 같았다.
\begin{center}
    \begin{tabular}{l|r|r|r|r|r}
        각 레벨에 남긴 노드 수의 상한 & 2* & 4 & 8 & 16 & 32 \\
        \hline
        점수 평균 & 126,047 & 130,127 & 129,551 & 132,435 & 130,272 \\
        점수 표준편차 & 7,210 & 4,449 & 5,871 & 8,672 & 7,253 \\
        \hline
        누적 계산 시간 평균 (초) & 0.17 & 1.00 & 1.89 & 3.86 & 7.26 \\
        누적 계산 시간 표준편차 (초) & 0.07 & 0.15 & 0.07 & 0.08 & 0.25 \\
        1초당 점수 & 741,453 & 130,127 & 68,546 & 34,310 & 17,944 \\
        \hline
        누적 사용 메모리 평균 (MB) & 35.32 & 64.68 & 120.70 & 227.33 & 397.50 \\
        누적 사용 메모리 표준편차 (MB) & 0.67 & 0.97 & 1.64 & 3.06 & 4.97 \\
        1MB당 점수 & 3,569 & 2,012 & 1,073 & 583 & 324 \\
    \end{tabular}
\end{center}
* 게임 오버를 당하는 경우는 제외하였다.

의외로 노드 수의 상한과 점수 평균에는 상관관계가 없음을 확인할 수 있다. 다만, 노드를 2개
남겼을 때에는 1,000번째 블록이 떨어지기 전에 게임 오버를 당하는 경우가 있었다. 이 요인도
안정성과 효율의 균형이 맞는 값을 잘 선택해야 할 것이다.

이 평가를 하기 직전까지는 프로젝트에서 레벨마다 확인하는 노드의 상한이 32개였는데,
평가 이후 5개로 감소시켰다.

누적 계산 시간 평균의 그래프는 다음과 같았다.

\begin{center}
    \begin{tikzpicture}[domain=0:33]
        \begin{axis}[xlabel={각 레벨에 남긴 노드 수의 상한}, ylabel={누적 계산 시간(초)}]
        \addplot[scatter, scatter src=\thisrow{class},
              error bars/.cd, y dir=both, x dir=both, y explicit, x explicit, error bar style={color=mapped color}]
              table[x=x,y=y,x error=xerr,y error=yerr] {
            x  xerr y yerr class
            2  0 0.17 0.07 0
            4  0 1.00 0.15 0
            8  0 1.89 0.07 0
            16 0 3.86 0.08 0
            32 0 7.26 0.25 0
        };
        \end{axis}
    \end{tikzpicture}
\end{center}

선형의 경향성을 보였다.

누적 사용 메모리 평균의 그래프는 다음과 같았다.

\begin{center}
    \begin{tikzpicture}[domain=0:33]
        \begin{axis}[xlabel={각 레벨에 남긴 노드 수의 상한}, ylabel={누적 사용 메모리(MB)}]
        \addplot[scatter, scatter src=\thisrow{class},
              error bars/.cd, y dir=both, x dir=both, y explicit, x explicit, error bar style={color=mapped color}]
              table[x=x,y=y,x error=xerr,y error=yerr] {
            x xerr y yerr class
            2  0 35.32 0.67 0
            4  0 64.68 0.97 0
            8  0 120.70 1.64 0
            16 0 227.33 3.06 0
            32 0 397.50 4.97 0
        };
        \end{axis}
    \end{tikzpicture}
\end{center}

마찬가지로 선형의 경향성을 보였다. 이는 추천 알고리즘의 공간 복잡도가 가지치기
상한선에 비례함을 뒷받침한다.

마지막으로 1초당 점수의 그래프는 다음과 같았다.

\begin{center}
    \begin{tikzpicture}[domain=0:33]
        \begin{axis}[xlabel={각 레벨에 남긴 노드 수의 상한}, ylabel={1초당 점수}]
        \addplot[scatter, scatter src=\thisrow{class},
              error bars/.cd, y dir=both, x dir=both, y explicit, x explicit, error bar style={color=mapped color}]
              table[x=x,y=y,x error=xerr,y error=yerr] {
            x xerr y yerr class
            2  0 741453 0 0
            4  0 130127 0 0
            8  0 68546 0 0
            16 0 34310 0 0
            32 0 17944 0 0
        };
        \end{axis}
    \end{tikzpicture}
\end{center}

노드 수의 상한과 1초당 점수는 반비례하는 경향을 보였다. 이는 남긴 노드 수와
관계없이 얻은 점수 평균이 거의 일정한 것에서도 확인할 수 있다.