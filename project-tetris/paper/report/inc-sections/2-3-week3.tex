\subsubsection{3주차: 추천 위치를 계산하는 로직} 추천 위치를 계산하는 로직은 다음과 같이 세분화된다.
\begin{itemize}
    \item 결정 트리를 만드는 로직
    \item 보드 상태 점수를 계산하는 로직
    \item 결정 트리를 최적화하는 로직
\end{itemize}

현재 상태를 루트라고 하면, 다음 $n$개 블록에 대한 높이 $n+1$의 결정 트리를 만들 수 있다. 결정 트리의 각 노드에는
보드 상태와 점수가 포함된다.

공간 복잡도의 개선을 위해 탐색과 가지치기를 진행하면서 결정 트리를 만들어나간다. 가지치기를 하기 위해서는
노드를 정렬할 필요가 있다. 아래는 $n$개의 노드를 amortized $\mathcal{O}\left(n \log n\right)$의
시간에 정렬하는 메서드이다. Swap의 의사 코드는 trivial하므로 기술하지 않는다.

\begin{codebox}
\Procname{$\proc{Quicksort}(\id{array},\,\id{start},\,\id{end})$}
\li \If $\id{start} \geq \id{end}$ \Then
\li     \Return
    \End
\li \Comment Set pivot to first element
\li $i \gets \id{start}$, $j \gets \id{end}$, $p \gets \id{start}$
\li \While $i < j$ \Do
\li     \While $\id{array}[i] \leq \id{array}[p]$ and $i < \id{end}$ \Do
\li         $i \gets i + 1$
        \End
\li     \While $\id{array}[j] > \id{array}[p]$ \Do
\li         $j \gets j + 1$
        \End
\li     \If $i<j$ \Then
\li         Swap $\id{array}[i]$ and $\id{array}[j]$
        \End
    \End
\li Swap $\id{array}[p]$ and $\id{array}[j]$
\li $\proc{Quicksort}(\id{array},\,\id{start},\,\id{j} - 1)$
\li $\proc{Quicksort}(\id{array},\,\id{j} + 1,\,\id{end})$
\end{codebox}

\begin{wrapfigure}[8]{l}{0.4\textwidth}
    \OMINO{%
    .............\\%
    .............\\%
    +-----+......\\%
    |X.X.X|......\\%
    |.....+.+-+..\\%
    |X.X.X|.|T|..\\%
    |...+-+-+.+-+\\%
    |X.X|.|T.T.T|\\%
    +---+.+-----+\\%
    }
    \caption{비효율적으로 배치된 T 블록}
    \label{fig:inefficient-t}
\end{wrapfigure}

결정 트리를 만들 때, 블록을 놓을 수 있는 위치를 모두 계산해야 한다. 단순히 각 $x$좌표마다
블록이 놓일 수 있는 최대 $y$좌표를 계산하는 것은 Figure \ref{fig:inefficient-t}와
같은 상황에서 T 블록을 왼쪽으로 끼워넣어 효율적인 상황을 만들 수 있음에도 불구하고 끼워넣지 않는
비효율적인 상황만을 탐색하게 되므로, 이런 상황과 같이 조각을 밀어넣어 필드를 더 효율적으로 만들
수 있는 경우도 놓치지 않고 탐색하기 위해 BFS를 사용한다.

\begin{codebox}
\Procname{$\proc{FindAvailableSpots}(\id{field},\,\id{nextBlock})$}
\li Allocate new queue $q$, $\id{avilable\_spots}$
\li \For $\id{rot} \gets 0$ \To $4$ \Do
\li     \For $x \gets -4$ \To $\id{WIDTH} + 4$ \Do
\li         \If $\proc{CheckToMove}(\id{field}, \id{nextBlock}, \id{rot}, 1, x)) \isequal \const{true}$ \Then
\li             Emplace $(5, x + 4, \id{rot})$ to $q$
            \End
        \End
    \End
\li \While $\attrib{q}{size} > 0$ \Do
\li     $(y + 4,\,x + 4,\,r) \gets \attrib{q}{front}$
\li     Pop $q$
\li     \For all neighboring cell $(\id{ny},\,\id{nx})$ to $(y,\,x)$ \Do
\li         \For current rotation state and next rotation state $\id{nr}$ \Do
\li             \If state $(\id{ny},\,\id{nx},\,\id{nr})$ not checked \Do
\li                 \If $\proc{CheckToMove}(\id{field}, \id{nextBlock}, \id{nr}, \id{ny}, \id{nx})) \isequal \const{true}$ \Then
\li                     \If $\id{ny} \isequal y + 1$ and $\id{nr} \isequal r$ \Do
\li                         Emplace $(y + 4,\,x + 4,\,r)$ to $\id{avilable\_spots}$
\li                     \ElseNoIf
\li                         Emplace $(\id{ny} + 4,\,\id{nx} + 4,\,\id{nr})$ to $q$
\li                     \End
                    \End
                \End
            \End
        \End
    \End
\li \Return $\id{avilable\_spots}$
\end{codebox}

위의 메서드는 블록을 필드 또는 바닥에 닿게 하면서 사용자가 키보드를 조작해 블록을 놓을 수 있는 모든
위치를 BFS로 찾아 준다. 전체 필드 공간을 3-tuple을 이용한 BFS로 탐색하면서
\proc{CheckToMove}를 호출하므로 시간 복잡도는 $\mathcal{O}\left(WIDTH
\times HEIGHT\right)$인데, 이는 각 $x$좌표마다 블록이 놓일 수 있는 최대 $y$좌표를 계산하는
시간 복잡도와 동일하다.

결정 트리를 만드는 것과 최적화하는 것은 동시에 이루어진다. \proc{Recommend-BFS}는 한 레벨의
노드들의 리스트를 받아 다음 노드들을 만든 후, 정렬하고, 상위 몇 개 노드들을 다시 \proc{Recommend-BFS}로
보내 주는 재귀적 메서드이다.

이 메서드는 끝 레벨이 아닐 경우 매번 호출될 때마다 \id{nodes}의 크기만큼 \proc{FindAvailableSpots}를 호출하며
$(\id{nodes} \times \id{children})$의 크기만큼 \proc{Board-Score}를 호출한다. \id{nodes}는 위의 레벨에서 가지치기 상한선 개만큼 들어올 수 있으며
\proc{FindAvailableSpots}의 공간 복잡도는 $\mathcal{O}\left(WIDTH \times HEIGHT\right)$이므로
\id{children}의 공간 복잡도도 그러하고, \proc{Board-Score}의 시간 복잡도는 $\mathcal{O}\left(WIDTH \times HEIGHT^2\right)$이므로 
노드 하나에 대해 \id{childs}를 만드는 시간은 $\mathcal{O}\left(WIDTH^2 \times HEIGHT^3\right)$이다.

한편 가지치기 상한선을 $m$이라 하면, 생길 수 있는 \id{childs}의 개수는 \id{node}마다
$\mathcal{O}\left(WIDTH \times HEIGHT\right)$개씩이고 이후에
amortized $\mathcal{O}\left(n \log n\right)$의 시간에 정렬되므로, 이를 수행하는 데 걸리는 시간은
\[\mathcal{O}\left(m \times WIDTH \times HEIGHT \times \log \left(m \times WIDTH \times HEIGHT\right)\right)\]
이다. 따라서 이 메서드가 한 번 호출되려면
\[\mathcal{O}\left(m \times WIDTH^2 \times HEIGHT^3 + m \times WIDTH \times HEIGHT \times \log \left(m \times WIDTH \times HEIGHT\right)\right)\]
의 시간이 필요하며, 이는 $\mathcal{O}\left(m \times WIDTH^2 \times HEIGHT^3\right)$와 같다.

이 작업을 트리의 높이만큼 재귀적으로 반복하므로 다음 블록의 개수를 $k$라 하면 전체 시간 복잡도는
\[\mathcal{O}\left(k \times m \times WIDTH^2 \times HEIGHT^3\right)\]
이다.

\begin{codebox}
\Procname{$\proc{Recommend-BFS}(\id{level},\,\id{nodes})$}
\li \If $\id{level} \isequal \mbox{number of future blocks}$ \Then
\li     $\id{max\_score} \gets -\infty$
\li     $\id{max\_node} \gets \const{null}$
\li     \For $\id{node}$ in $\id{nodes}$ \Do
\li         \If $\id{max\_score} > \attrib{node}{score}$ \Then
\li             $\id{max\_score} \gets \attrib{node}{score}$
\li             $\id{max\_node} \gets \id{node}$
            \End
        \End
\li     \If $\id{max\_node} \isequal \const{null}$ \Then
\li         \Return \const{null}
        \End
\li     \While $\attrib{max\_node}{parent} \neq \const{null}$ and $\attribb{max\_node}{parent}{parent} \neq \const{null}$ \Then
\li         $\id{max\_node} \gets \attrib{max\_node}{parent}$
        \End
\li     \Return \id{max\_node}
    \End
\li Allocate new list \id{childs}
\li \For $\id{node}$ in $\id{nodes}$ \Do
\li     $\id{children} \gets \proc{FindAvailableSpots}(\attrib{node}{field},\,\id{nextBlock}[\id{level}])$
\li     \While $\attrib{children}{size} > 0$ \Do
\li         Allocate new node \id{newChild}
\li         Copy \attrib{node}{field} to \attrib{newChild}{field}
\li         $\attrib{newChild}{curBlockID} \gets \id{nextBlock}[\id{level}]$
\li         $(y + 4,\,x + 4,\,r) \gets \attrib{children}{front}$
\li         $\attrib{newChild}{recBlockX} \gets x$, $\attrib{newChild}{recBlockY} \gets y$, $\attrib{newChild}{recBlockRotate} \gets r$
\li         Pop \id{children}
\li         $\id{board\_score} \gets \proc{Board-Score}(\attrib{newChild}{field},\,\id{nextBlock}[\id{level}])$
\li         $\attrib{newChild}{parent} \gets \id{node}$, $\attrib{newChild}{score} \gets \id{board\_score}$
\li         Add \id{newChild} to \id{childs}
        \End
    \End
\li $\proc{Quicksort}(\id{childs},\,0,\,\mbox{count of }\id{childs}-1)$
\li $\id{pruned\_size} \gets \min(\mbox{count of }\id{childs},\,\mbox{tree prune limit})$
\li Allocate new list \id{pruned}
\li \For $i \gets 0$ \To \id{pruned\_size} \Do
\li     Add $\id{childs}[i]$ to \id{pruned}
    \End
\li \Return $\proc{Recommend-BFS}(\id{level} + 1,\,\id{pruned})$
\end{codebox}

보드 점수 계산은 \textbf{2.2}에서 언급한 18개의 휴리스틱들로 이루어진다. 보드 점수 $c$는 각각 산출된 18개의 휴리스틱 값 벡터 $\mathbf{h}$에
18차원 계수 벡터 $\mathbf{x}$를 내적해 구할 수 있다. 식으로 나타낼 경우
\[c = \mathbf{h} \cdot \mathbf{x}\]
이다. 각각의 휴리스틱을 계산하는 방법은 간단하므로 생략한다.

\begin{codebox}
\Procname{$\proc{Board-Score}(\id{field},\,\id{nextBlock},\,\id{rot},\,\id{blockY},\,\id{blockX})$}
\li Calculate heuristic value vector $\mathbf{h}$
\li \Return $\mathbf{h} \cdot \mathbf{x}$
\end{codebox}

이 메서드는 줄을 지우는 부분에서 \proc{DeleteLine}과 같은
$\mathcal{O}\left(WIDTH \times HEIGHT^2\right)$번의 동작을 하며, 다른 휴리스틱을 구하는 부분에서는
최대 $\mathcal{O}\left(WIDTH \times HEIGHT\right)$번의 루프를 돈다. 따라서 전체 시간 복잡도는
$\mathcal{O}\left(WIDTH \times HEIGHT^2\right)$이다.

\subsubsection{3주치: 컴퓨터가 자동으로 게임을 진행하는 로직} 자동 플레이 플래그를 만들고,
플래그가 설정되어 있을 경우 프로그램의 실행 동작을 바꾸도록 한다. 가령 자동 플레이 플래그가
설정되어 있고 현재 추천 위치에 블록을 놓을 수 있을 경우 \proc{BlockDown}에서 블록을
추천 위치에 고정시며, 사용자의 입력을 모두 무시한다.

\subsubsection{3주치: 추천 알고리즘을 개선하는 프로그램} 추천 알고리즘을 효과적으로 개선하기 위해
\textbf{2.2}에서 소개한 담금질 기법을 사용한다. $T=1000$, $k=0.01$로 설정하며 $T<10^{-3}$일 경우
충분히 최적해에 수렴했다고 보고 훈련을 종료한다. 비용 함수는 위에서 계산한 보드 점수이다.

프로그램을 실행하고 게임 오버가 되기를 기다린 뒤 결과를 직접 수집하는 것은 효율적인 방법이 아니다. 보유하고 있는
장비들은 다수의 CPU 코어를 사용할 수 있는 경우가 많은데, 프로그램을 동시에 실행할 수 있디면 컴퓨터의 자원을
최대한 활용할 수 있아 좋을 것이다. 또한 결과가 나오기까지 직접 기다리지 않고 추천 알고리즘을 개선하는 프로그램이
결과를 수집해서 통계까지 산출해 준다면 이상적일 것이다.

이를 위해 JNI\translation{Java Native Interface}를 사용한다. JNI는 JVM\translation{Java Virtual Machine}
위에서 실행되는 Java 프로그램이 네이티브 라이브러리를 참조할 수 있게 만든 만들어진 인터페이스이다.

C에서 멀티스레딩을 구현하는 것은 어렵지만 이미 존재하는 C 코드를 JNI가 사용할 수 있는 라이브러리로 포팅하는
것은 비교적 쉬우므로, 우선 ncurses.h에 의존하는 기능을 전부 지우고 \texttt{main} 함수를 계수 벡터
$\mathbf{x}$를 받아 추천 알고리즘을 시뮬레이션하게 만든다. 그리고 게임 오버 직전에 떨어뜨린 블록 수,
점수, 블록 당 점수 등의 정보를 반환해 Java 프로그램에서 이를 받아 $\mathbf{x}$ 값에 반영하도록 한다.

Java는 여러 스레드\translation{thread}를 생성해 동시에 여러 CPU 코어를 사용하기 쉽게 만들어진 언어이므로
이와 같은 프로그램을 효율적으로 작성할 수 있다. 또한 Java는 플랫폼 독립적이기 때문에 macOS에서 빌드한 .jar 파일을
Linux에서 그대로 실행시킬 수 있다. 다만, JNI에서 사용되는 네이티브 라이브러리는 macOS에서는
.dylib, Linux에서는 .so와 같은 식으로 플랫폼 종속적이므로 따로 빌드해 줘야 한다.