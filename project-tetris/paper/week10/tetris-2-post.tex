\documentclass[runningheads]{../../../llncs}
\usepackage[paperheight=295mm,paperwidth=210mm]{geometry}
\usepackage{graphicx}
\usepackage{import}
\usepackage{kotex}
\usepackage[dvipsnames]{xcolor}
\usepackage{fancyvrb}
\usepackage{listings}
\usepackage{indentfirst}
\usepackage{tabularx}
\usepackage{underscore}
\usepackage{multicol}
\usepackage{enumitem}
\usepackage{menukeys}
\usepackage{amsmath}
\usepackage{clrscode3e} % https://www.ctan.org/pkg/clrscode3e?lang=en
\usepackage[numbers,square,super]{natbib}
\usepackage{inconsolata} % Inconsolata
\usepackage{mathptmx} % Times New Roman
\usepackage{minted}
\graphicspath{ {./images/} }
\lstset{basicstyle=\footnotesize\ttfamily,breaklines=true}
\renewcommand{\bibname}{참고문헌}
\setlength{\parindent}{1em}
\setlength{\parskip}{1em}
\linespread{1.2}
{\renewcommand{\arraystretch}{1.5}%
\setlength{\tabcolsep}{0.5em}%
\newenvironment{Figure}
  {\par\medskip\noindent\minipage{\linewidth}}
  {\endminipage\par\medskip}
\newcommand{\translation}[1]{\textsuperscript{#1}}
\newlist{algorithm}{enumerate}{10}
\setlist[algorithm]{label*=\arabic*.}
\setlist[algorithm,1]{label=\textbf{\arabic*}}
\setlist[algorithm,2]{label=\textbf{\alph*}}
\setlist[algorithm,3]{label=\textbf{\roman*}}
\setlist[algorithm,4]{label=(\arabic*)}
\setlist[algorithm,5]{label=(\alph*)}
\setlist[algorithm,6]{label=(\roman*)}
\makeatletter
\renewcommand\NAT@citesuper[3]{\ifNAT@swa
\if*#2*\else#2\NAT@spacechar\fi
\unskip\kern\p@\textsuperscript{\NAT@@open#1\if*#3*\else,\NAT@spacechar#3\fi\NAT@@close}%
   \else #1\fi\endgroup}
\makeatother
    
\usepackage{subfig}

\begin{document}

\title{CSE3013 (컴퓨터공학 설계 및 실험 I) \space \newline PRJ-2 테트리스 프로젝트 2주차 결과 보고서}
\author{서강대학교 컴퓨터공학과 박수현 (20181634)}
\institute{서강대학교 컴퓨터공학과}
\maketitle

\section{목적}
테트리스 게임에 랭킹을 추가한다.

\section{문제}
알고리즘의 의사 코드는 예비 보고서에서 작성한 코드와 동일하다. 실제 소스 코드에서는 메서드 이름에 BST\translation{Binary Search Tree}나 RB-tree 등의 
이름을 사용하지 않고 ordered list라는 표현을 사용했는데, 이는 자료 구조의 동작을 추상화하고 main.c에서 조금 더 이해하기 쉬운 코드를 작성할 수 있게 하려는 일환이다. 

\newpage

다음은 실습에서 구현한 함수들의 기능과 복잡도 일람이다. 

\begin{tabularx}{\textwidth}{l|X|l|l}
    함수명 & 기능 & 시간 복잡도 & 공간 복잡도 \\
    \hline
    \proc{BST-Left-Rotate} & 노드 $x$의 위치로 $x$의 오른쪽 자식을 옮긴다. & $\mathcal{O}\left(1\right)$ & $\mathcal{O}\left(1\right)$ \\
    \proc{BST-Right-Rotate} & 노드 $x$의 위치로 $x$의 왼쪽 자식을 옮긴다. & $\mathcal{O}\left(1\right)$ & $\mathcal{O}\left(1\right)$ \\
    \proc{BST-Transplant} & $u$의 서브트리를 $v$의 서브트리로 대체한다. & $\mathcal{O}\left(1\right)$ & $\mathcal{O}\left(1\right)$ \\
    \proc{BST-Get} & 특정 순위의 노드를 가져온다. & $\mathcal{O}\left(\log n\right)$ & $\mathcal{O}\left(n\right)$ \\
    \hline
    \proc{BST-Insert} & 트리에 새 노드를 삽입한다. & $\mathcal{O}\left(\log n\right)$ & $\mathcal{O}\left(1\right)$ \\
    \proc{BST-Insert-Revalidate} & 트리에 새 노드를 삽입한 후, RB-tree의 조건을 충족하도록 트리를 고친다. & $\mathcal{O}\left(\log n\right)$ & $\mathcal{O}\left(1\right)$ \\
    \proc{BST-Delete} & 트리에서 노드를 삭제한다. & $\mathcal{O}\left(\log n\right)$ & $\mathcal{O}\left(1\right)$ \\
    \proc{BST-Delete-Revalidate} & 트리에서 노드를 삭제한 후, RB-tree의 조건을 충족하도록 트리를 고친다. & $\mathcal{O}\left(\log n\right)$ & $\mathcal{O}\left(1\right)$ \\
    \hline
    \proc{BST-Query} & 특정 순위 구간에 있는 노드들을 쿼리한다. & $\mathcal{O}\left(n\right)$ & $\mathcal{O}\left(n\right)$ \\
\end{tabularx}

예비 보고서에서 \proc{BST-Query}에 대한 $\mathcal{O}\left(\left(r - l\right) \log n\right)$ 최적화를 언급했는데,
실제 코드에서 게임이 끝난 후 최대 5개까지의 원소만을 쿼리했다. 랭킹 리스트가 커지고 이 작업이 자주 일어난다면, 5개의 원소만을 쿼리할 때
$r - l = 5$이고 이 때 이 메서드의 시간 복잡도는 $\mathcal{O}\left(\log n\right)$이 되고, 이는 $n$이 충분히 클 경우 일반적으로
$\mathcal{O}\left(n\right)$보다 효율적이므로 최적화가 수행된 메서드를 따로 만드는 것도 고려할 수 있다.

다음은 각 연산에 대한 정렬 상태의 연결 리스트와 RB 트리로 구현된 BST의 복잡도 비교 일람이다.

\begin{tabularx}{\textwidth}{X|l|l|l|l}
    기능 & \multicolumn{2}{l|}{정렬 상태의 연결 리스트} & \multicolumn{2}{l}{BST} \\
     & 시간 복잡도 & 공간 복잡도 & 시간 복잡도 & 공간 복잡도 \\
    \hline
    원소 삽입 & $\mathcal{O}\left(n\right)$ & $\mathcal{O}\left(1\right)$ & $\mathcal{O}\left(\log n\right)$ & $\mathcal{O}\left(1\right)$ \\
    원소 삭제 & $\mathcal{O}\left(n\right)$ & $\mathcal{O}\left(1\right)$ & $\mathcal{O}\left(\log n\right)$ & $\mathcal{O}\left(1\right)$ \\
    쿼리 & $\mathcal{O}\left(n\right)$ & $\mathcal{O}\left(1\right)$ & $\mathcal{O}\left(n\right)$ & $\mathcal{O}\left(n\right)$ \\
\end{tabularx}

따라서 쿼리할 때의 공간 복잡도를 제외하고 모든 면에서 일반적으로 BST가 정렬 상태의 연결 리스트보다 효율적이다.

\section{습득한 내용}
C++ STL의 \texttt{std::set}을 사용해 본 적은 많았으나 내부 원리를 자세하게 이해하고 구현할 수 있는 정도는 아니었는데,
\texttt{std::set}의 내부적인 원리와 RB 트리를 \textit{Introduction to Algorithms}을 참고해 이해하고 직접 구현해 볼 수 있는 기회가 되었다.

\begin{thebibliography}{1}
    \bibitem{CLRS} T. H. Cormen, C. E. Leiserson, R. L. Rivest, C. Stein, \textit{Introduction
    to Algorithms}. Cambridge, MA: The MIT Press, 2009, pp. 287-338.
\end{thebibliography}

\end{document}
