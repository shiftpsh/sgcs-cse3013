
\documentclass[runningheads]{llncs}
\usepackage[paperheight=295mm,paperwidth=210mm]{geometry}
\usepackage{graphicx}
\usepackage{kotex}
\usepackage[dvipsnames]{xcolor}
\usepackage{fancyvrb}
\usepackage{listings}
\usepackage{indentfirst}
\usepackage{tabularx}
\usepackage{underscore}
\usepackage{multicol}
\usepackage[square,sort,comma,super]{natbib}
\usepackage{inconsolata} % Inconsolata
\usepackage{mathptmx} % Times New Roman
\usepackage[cache=false]{minted}
\graphicspath{ {./images/} }
\lstset{basicstyle=\footnotesize\ttfamily,breaklines=true}
\renewcommand{\bibname}{참고문헌}
\setlength{\parindent}{1em}
\setlength{\parskip}{1em}
\linespread{1.2}
{\renewcommand{\arraystretch}{1.5}%
\setlength{\tabcolsep}{0.5em}%
\newenvironment{Figure}
  {\par\medskip\noindent\minipage{\linewidth}}
  {\endminipage\par\medskip}
	
\begin{document}

\title{CSE3013 (컴퓨터공학 설계 및 실험 I) \space \newline UNIX-1 예비 보고서}
\author{서강대학교 컴퓨터공학과 박수현 (20181634)}
\institute{서강대학교 컴퓨터공학과}
\maketitle

\section{목적}
UNIX 시스템에 대하여 미리 접해본 후 실험에 임할 수 있도록 한다.

\section{예비 학습}
cspro에 처음 접속하면 홈 디렉토리는 다음과 같다.
\begin{minted}[xleftmargin=\parindent,breaklines,linenos]{shell}
cse20181634@cspro:~$ pwd
/sogang/under/cse20181634
\end{minted}

실험에 사용될 주소록 데이터를 다음과 같이 작성했다.
\begin{minted}[xleftmargin=\parindent,breaklines,linenos]{text}
문재인|서울특별시 종로구 세종로 1|02-730-5800
박원순|서울특별시 중구 세종대로 110|02-120
홍길동|서울시 마포구 신수동 서강대학교 AS관 301호|02-705-2665
박문수|서울시 서대문구 신촌동 연세대학교 연구관 102호|02-708-4678
Andrew|경기도 의정부시 호원동 23-12번지|031-827-7842
\end{minted}

작성하는 데 다음과 같은 명령어가 사용되었다.

\begin{tabular}{l|l}
	명령어 & 수행 결과 \\
	\hline
	\texttt{vi data} (shell) & vi 에디터를 실행한다. \\
	\texttt{i} (vi) & vi 에디터를 INSERT 모드로 전환한다. \\
	\texttt{:!wq} (vi) & 파일을 저장하고 에디터를 종료한다.
\end{tabular}

데이터 생성 후 파일을 \$home/.data로 복사했다.

\begin{minted}[xleftmargin=\parindent,breaklines,linenos]{shell}
cse20181634@cspro:~$ cp data $HOME/.data
\end{minted}

그룹 및 다른 사용자가 권한을 갖지 않도록 변경했다. chmod 명령어에서 권한은 3자리 8진수로 표현할 수 있는데, 첫 자리는 user, 두 번째 자리는 group, 세 번째 자리는 other를 의미한다. 또한 각 자리는 read, write, execute를 각각 1, 2, 4로 두고 이 플래그들의 bitwise OR로 권한을 표현한다. 파일 소유자만 전권을 갖고 다른 계정에 대해서는 권한을 부여하지 않고 싶다면, 이 때 해당되는 권한은 700이다.
\begin{minted}[xleftmargin=\parindent,breaklines,linenos]{shell}
cse20181634@cspro:~$ chmod 700 $HOME/.data
\end{minted}

또한 +r, +w, +x 등으로 권한을 설정할 수도 있는데, 각각이 의미하는 바는 다음과 같다.

\begin{tabularx}{\textwidth}{l|X}
	권한 & 의미 \\
	\hline
	\texttt{+r} & read의 머리글자이다. 파일을 읽을 수 있는 권한이다. 디렉토리에 설정된 경우 디렉토리 안의 파일 목록을 조회할 수 있게 된다. \\
	\texttt{+w} & write의 머리글자이다. 파일을 작성, 수정, 삭제할 수 있는 권한이다. 디렉토리에 설정된 경우 디렉토리 안에 새로운 파일이나 디렉토리를 만들 수 있게 된다. 또한 현재 디렉토리를 수정하는 것은 가능해지나 삭제하는 것은 하위 파일 및 디렉토리에 모두 write 권한이 있어야 가능하다. \\
	\texttt{+x} & execute의 두 번째 글자이다. 파일을 실행할 수 있는 권한이다. 디렉토리에는 이 권한이 없으면 read, write 전부 불가능하게 된다.
\end{tabularx}

\section{보충 학습}
정규 표현식(regular expression) 또는 정규식은 어떤 특정한 규칙을 가진 문자열의 집합을 표현하는 데 사용되는 언어이다. 간단히 regexp 또는 regex라고도 한다. 정규 표현식은 검색 엔진, 워드 프로세서, sed 또는 awk과 같은 문자열 처리 유틸리티에서 쓰인다. 많은 프로그래밍 언어는 정규식 처리를 지원한다.

정규식은 여러 형식이 있었는데, 현재는 IEEE POSIX 표준을 주로 사용하고 있다. 전술했듯이 각각의 정규식은 하나의 집합인데, 예를 들어 정규식 \texttt{.at}는 \texttt{cat}, \texttt{bat}, \texttt{hat}을 포함하고 \texttt{s.*}는 \texttt{s}로 시작하는 모든 글자들을 포함한다. 여기서 정규식에 사용되는 특수 문자를 메타캐릭터(metacharacters)라고 한다. POSIX 표준에서 정의된 메타캐릭터의 예는 다음과 같다.

\begin{tabularx}{\textwidth}{l|X}
	메타캐릭터 & 의미 \\
	\hline
	\texttt{\textasciicircum} & 줄의 시작에 있는 문자열로 집합을 한정 \\
	\texttt{.} & 문자 하나 \\
	\texttt{[ ]} & 괄호 안의 문자열 중 하나 \\
	\texttt{[\textasciicircum\space]} & 괄호 안의 문자열을 제외한 문자열 중 하나 \\
	\texttt{\$} & 줄의 끝에 있는 문자열로 집합을 한정 \\
	\texttt{( )} & 부분 표현식 정의 \\
	\texttt{x*} & \texttt{x}가 0번 이상 등장하는 문자열 \\
	\texttt{x+} & \texttt{x}가 1번 이상 등장하는 문자열 \\
	\texttt{x?} & \texttt{x}가 0번이나 1번만 등장하는 문자열 \\
\end{tabularx}

\end{document}
